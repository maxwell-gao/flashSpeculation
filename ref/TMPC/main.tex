
\documentclass{article} %
\usepackage{iclr2026_conference}
% math_commands.tex — Shorthand macros for the SignFlip paper

% Greek shortcuts
\newcommand{\al}{\alpha}
\newcommand{\bt}{\beta}

% Model components
\newcommand{\lmhead}{\texttt{lm\_head}}
\newcommand{\hN}{h_N}          % final hidden state
\newcommand{\hL}{h_L}          % intermediate hidden state
\newcommand{\Ddeep}{\Delta_{\text{deep}}}  % residual h_N - h_L

% Distributions
\newcommand{\pguided}{p_{\text{guided}}}
\newcommand{\ptarget}{p_{\text{target}}}
\newcommand{\pproposal}{q_{\text{proposal}}}

% Operators
\newcommand{\softmax}{\operatorname{softmax}}
\newcommand{\logsoftmax}{\operatorname{log\text{-}softmax}}

% Placeholder for results not yet available
\newcommand{\placeholder}[1]{\textcolor{red}{\textbf{[#1]}}}

% Condition names (monospace in text)
\newcommand{\cond}[1]{\texttt{#1}}


\usepackage{hyperref}
\usepackage{url}

\usepackage{times}
\usepackage{latexsym}
\usepackage{microtype}
\usepackage{inconsolata}
\usepackage{graphicx}
\usepackage[table,xcdraw]{xcolor}
\usepackage{algorithm}
\usepackage{algorithmic}
\usepackage{multirow} 
\usepackage{arydshln}
\usepackage{amsmath} 
\usepackage{amssymb} 
\usepackage{cleveref} 
\usepackage{comment}
\usepackage{wrapfig}
\usepackage{booktabs}
\newcommand{\ie}{\textit{i.e., }}
\newcommand{\eg}{\textit{e.g., }}
\newcommand{\pchtext}[1]{\textbf{\textcolor{teal}{#1}}}
\newcommand{\pch}[1]{{\textcolor{teal}{#1}}}
\newcommand{\rebu}[1]{{\textcolor{blue}{#1}}}
\newcommand{\gdtext}[1]{{\textcolor{brown}{#1}}}
\newcommand{\yhtext}[1]{\textcolor{purple}{#1}}
\definecolor{lightyellow}{HTML}{FFFFC7}
\usepackage{subcaption} 
\usepackage{pgfplots}
\pgfplotsset{compat=1.18} 


\title{Test-Time Alignment for Large Language Models via Textual Model Predictive Control}



\author{
\textbf{Kuang-Da Wang}\textsuperscript{1,†} 
\textbf{Teng-Ruei Chen}\textsuperscript{1,†}
\textbf{Yu-Heng Hung}\textsuperscript{1}
\textbf{Guo-Xun Ko}\textsuperscript{1} 
\textbf{Shuoyang Ding}\textsuperscript{2} \\
\ \textbf{Yueh-Hua Wu}\textsuperscript{2}
\textbf{Yu-Chiang Frank Wang}\textsuperscript{2}
\textbf{Chao-Han Huck Yang}\textsuperscript{2} \\
\ \textbf{Wen-Chih Peng}\textsuperscript{1}
\textbf{Ping-Chun Hsieh}\textsuperscript{1} \\
\ \textsuperscript{1}National Yang Ming Chiao Tung University, Hsinchu, Taiwan \quad
\textsuperscript{2}NVIDIA \\
\ \texttt{\{gdwang.cs10,pinghsieh\}@nycu.edu.tw, hucky@nvidia.com}
}

\hypersetup{hidelinks}




\newcommand{\fix}{\marginpar{FIX}}
\newcommand{\new}{\marginpar{NEW}}

\iclrfinalcopy %
\begin{document}


\maketitle

\footnotetext[1]{Equal contribution. Correspondence to Kuang-Da Wang \texttt{<gdwang.cs10@nycu.edu.tw>}, Ping-Chun Hsieh \texttt{<pinghsieh@nycu.edu.tw>}, and Chao-Han Huck Yang \texttt{<hucky@nvidia.com>}.}

\begin{abstract}
Aligning Large Language Models (LLMs) with human preferences through finetuning is resource-intensive, motivating lightweight alternatives at test time. We address test-time alignment through the lens of sequential decision making, a perspective that reveals two fundamental challenges. When actions are defined at the token level, as in guided decoding, alignment suffers from the \textit{curse of horizon}. Conversely, when actions are at the response level, as in traditional iterative refinement, the \textit{curse of dimensionality} emerges. To resolve this trade-off, we draw inspiration from Model Predictive Control (MPC) in control theory to propose \textbf{Textual Model Predictive Control (TMPC)}, a novel predictive planning framework adapted for aligning LLMs at inference time. A key limitation of standard MPC is its reliance on predefined, hard segment boundaries, which are often absent in text generation. TMPC overcomes this by introducing two principles inspired by hierarchical reinforcement learning: (1) \textit{Hindsight Subgoal Identification}, where TMPC analyzes generation subgoals to retrospectively identify high-reward intermediate outputs as subgoals. This allows the framework to discover meaningful, task-specific planning steps (\eg a sentence in machine translation or a bug fix in code generation.). (2) \textit{Subgoal-Conditioned Re-Generation}, where these identified subgoals are used to guide subsequent planning iterations. By conditioning on these proven, high-quality subgoals, TMPC ensures stable improvement by building upon previously validated successes. TMPC is evaluated on three tasks with distinct segmentation properties: discourse-level translation, long-form response generation, and program synthesis. The results demonstrate that TMPC consistently improves performance, highlighting the generality.
\end{abstract}

\section{Introduction}
\label{sec:intro}

The emergence of Large Language Models (LLMs), such as the GPT series~\citep{achiam2023gpt,brown2020language}, LLaMAs~\citep{touvron2023llama1,touvron2023llama}, and Gemma~\citep{team2024gemma}, has demonstrated remarkable efficacy in a wide range of NLP tasks \citep{hendrycks2021measuring,srivastava2023beyond,stiennon2020learning,yu2024kola,zhong-etal-2024-agieval}. While these models exhibit strong performance out of the box, aligning their outputs to human preferences remains critical, especially for smaller-scale LLMs. For instance, in machine translation~\citep{alves2024tower}, smaller LLMs (\eg under 10B parameters) frequently suffer from omissions and semantic drift~\citep{wu-etal-2024-word}. 
Thus, aligning LLM outputs to preferences remains an essential yet challenging problem.  

Training-time approaches such as Reinforcement Learning with Human Feedback (RLHF)~\citep{ouyang2022training} and Direct Preference Optimization (DPO)~\citep{rafailov2023direct} have achieved strong results in aligning preferences. However, these methods are resource-intensive and require costly retraining whenever preferences or tasks change. This has spurred interest in \textit{test-time alignment}, where outputs are adapted without updating model parameters, using strategies such as prompting~\citep{lin2024unlocking}, guided decoding~\citep{khanov2024args,re-control,li2024rain,wang-etal-2024-inferaligner,GenARM}, or iterative refinement~\citep{tpo}.  


\begin{figure*}[!t]
  \centering
  \includegraphics[width=\linewidth]{Concept.pdf}
  \caption{Textual Model Predictive Control (TMPC) balances the curse of horizon in guided decoding against the curse of dimensionality in naive iterative refinement. It employs Hindsight Subgoal Identification to dynamically discover promising states from  rollouts and Subgoal-Conditioned Re-Generation to guide the search from these discovered subgoals, ensuring a stable alignment.}
\label{fig:concept}
\end{figure*}

We address test-time alignment through the lens of sequential decision making, where the generation process is framed as a sequence of actions. This perspective reveals two fundamental challenges, illustrated in Figure~\ref{fig:concept} . When actions are defined at the token level (\eg guided decoding), methods suffer from the \textit{curse of horizon}~\citep{horizon}; credit assignment becomes unreliable over long trajectories, making alignment brittle. In contrast, when actions are at the response level (\eg iterative refinement), they face the \textit{curse of dimensionality}; each step involves rewriting an entire sequence, making the search for improvements in a vast action space intractable and unstable.

To address these challenges, we propose Textual Model Predictive Control (TMPC), a novel test-time alignment framework inspired by Model Predictive Control (MPC)~\citep{Camacho2007,kouvaritakis2016model}. 
While powerful, standard MPC assumes the problem can be decomposed into predefined, hard segment boundaries, a condition that rarely holds for complex text generation.
TMPC is uniquely adapted to overcome this limitation through two principles:
\begin{itemize}
\item \textbf{Hindsight Subgoal Identification:} This principle allows TMPC to discover meaningful planning steps. After generating candidate responses, TMPC retrospectively analyzes them to identify high-quality intermediate points as \textit{subgoals}. A subgoal can be a concrete unit, such as a sentence in translation, or an abstract one, such as resolving a single failed test case in program synthesis, 
successfully addressing the problem of lacking natural boundaries.
This hindsight-driven discovery effectively shortens the planning horizon for diverse tasks.
\item \textbf{Subgoal-Conditioned Re-Generation:} This principle ensures stable, cumulative progress. The subgoals identified via hindsight are stored in a buffer and used to guide subsequent planning iterations. By conditioning the next generation on these subgoals, TMPC ensures that subsequent generation builds upon these validated waypoints.
\end{itemize}
We evaluate TMPC on three challenging tasks with different boundary characteristics: WMT'24 discourse-level machine translation, the HH-RLHF long responses subset, and MBPP program synthesis. Experiments with LLaMA-3.1-8B-Instruct show that TMPC consistently improves alignment, highlighting the generality of our approach.

Our contributions are summarized as follows:
\begin{itemize}
\item We propose a novel formulation of test-time alignment as a sequential decision-making problem. This perspective unifies existing approaches and reveals a fundamental trade-off that governs their limitations: the \textit{curse of horizon} in guided decoding methods and the \textit{curse of dimensionality} in iterative refinement methods.
\item We introduce Textual Model Predictive Control (TMPC), a framework that adapts concepts from control theory to language generation. TMPC is operationalized through two principles: \textit{Hindsight Subgoal Identification} to discover subgoals from rollouts, and \textit{Subgoal-Conditioned Re-Generation} to iteratively improve generation by building on subgoals.
\item We empirically demonstrate the effectiveness of TMPC. TMPC achieves substantial improvements across three distinct domains including long-form response generation, discourse-level machine translation, and program synthesis, validating its ability to discover and leverage task-specific subgoals.
\end{itemize}

\section{Related Work}

\subsection{Preference Alignment Through Fine-tuning}
Aligning large language models (LLMs) with human preferences has traditionally relied on post-training strategies. Supervised fine-tuning (SFT) \citep{ziegler2019finetuning} and reinforcement learning from human feedback (RLHF) \citep{ouyang2022training} are widely used but computationally expensive. Direct Preference Optimization (DPO) \citep{rafailov2023direct} simplifies RLHF by converting it into a supervised learning objective, though it requires managing dual policies. More recent approaches like SimPO \citep{meng2024simpo} and Contrastive Preference Optimization (CPO)~\citep{xu2024contrastive} reduce memory and resource demands using reference models and contrastive signals. Despite these improvements, fine-tuning methods remain rigid and slow to adapt to changing data or objectives, posing challenges in dynamic environments.

\subsection{Test-Time Preference Alignment}
Test-time preference alignment offers an efficient way to align frozen language models by guiding generation at inference, without requiring any parameter updates. Beyond simple prompting or in-context learning, guided decoding methods harness external signals to control the generation itself. 
ARGS \citep{khanov2024args} is a representative example that incorporates reward model guidance at the token level, and InferAligner \citep{wang-etal-2024-inferaligner} adopts a similar strategy. Among guided decoding methods, there are also approaches that directly modify internal representations. For instance, RE-Control \citep{re-control} trains a value function on hidden states using the Bellman equation, and applies gradient-based optimization to align with preferences.
TreeBoN \citep{qiu2024treebon} and RAIN \citep{li2024rain} leverage tree-based structures: TreeBoN combines tree search with Best-of-N sampling, while RAIN performs self-evaluation without relying on a reward model to align preferences. 
GenARM~\citep{GenARM} enhances test-time alignment by introducing an autoregressive reward model that predicts next-token reward signals conditioned on prior context, enabling efficient, token-wise guidance that is theoretically expressive under a KL-regularized RL framework. Test-Time Preference Optimization (TPO)~\citep{tpo} takes a distinct approach, translating reward feedback into textual critiques that serve as language-based rewards. The model uses these to iteratively refine its output—effectively learning alignment on the fly.



\section{Background}
\label{sec:background}
In the general setup of RL for LLMs, text generation can be formally modeled as a finite-horizon Markov Decision Process (MDP). We adopt a general notion of a \textit{step} as the basic unit of temporal progression, which can represent a token, a segment at various granularities (\eg phrase, sentence, or paragraph), or other linguistically or structurally meaningful units. Then, an MDP can be defined as $\mathcal{M} = (\mathcal{S}, \mathcal{A}, \mathcal{P}, R, \mu, T)$, 
where (i) the state space \(\mathcal{S}\) consists of all possible text prefixes, (ii) the action space $\mathcal{A}$ corresponds to the set of all possible generation units, (iii) $\mathcal{P}$ denotes the transition function, (iv) $R:\mathcal{S}\times\mathcal{A}\rightarrow \mathbb{R}$ is the reward function that assigns scalar feedback to step-level or trajectory-level outcomes (\eg measuring fluency, factuality, or alignment with user preferences), (v) $\mu$ denotes the initial state distribution, and (vi) $T\in\mathbb{N}$ is the episode length.

We define the initial state $s_0$ as the initial prompt and let $a_t$ denote the partial response generated at step $t$. At each step $t$, the current state is the set of tokens from the initial prompt and the partial responses generated up to step $t$, \ie $s_t = (s_0,a_1, \cdots, a_{t-1})$.
Based on this construction, we know that the transition function is deterministic with $\mathcal{P}(s_{t+1} \rvert s_t, a_t) = 1$. A policy $\pi_\theta(a \rvert s)$, parameterized by the language model, defines a probability distribution over actions given the prefix $s\in\mathcal{S}$. 
The generation of a full text sequence of length $T$ can therefore be viewed as a trajectory $\tau=(s_0,a_0,\cdots,s_{T-1},a_{T-1},s_T)$ with the cumulative reward given by $\mathcal{J}(\tau):=\sum_{t=0}^{T-1} R(s_t, a_t)$. %

This perspective enables the application of RL methods to text generation in LLMs. Rather than relying solely on maximum likelihood estimation, which optimizes local token-level likelihoods, the MDP formulation allows optimization with respect to long-horizon objectives such as coherence and alignment with human preferences. This provides the foundation for recent advances in preference-based fine-tuning and test-time alignment.


\section{Methodology}
\label{sec:methodology}

\subsection{Test-Time Alignment via Trajectory Optimization}
\label{sec:TO}
Our key idea is to take a model-based RL viewpoint to achieve test-time alignment for LLMs. Specifically, we propose to recast preference alignment as \textit{trajectory optimization} and thereby employ receding-horizon control for iterative text generation.

\paragraph{Text Generation Optimization as Trajectory Optimization.} 
Usually adopted by the model-based RL literature~\citep{chua2018deep,lowrey2019plan}, the goal of trajectory optimization is to find an optimal sequence of actions $\boldsymbol{a}^*=(a_0^*,\cdots, a_{T-1}^*)$ such that the total trajectory-wise reward is maximized. This matches the objective of LLM text generation in that the output response is generated to best align with the underlying preference.
Recall from Section~\ref{sec:background} that we adopt a general notion of a \textit{step} as the basic unit of temporal progression, which can be a segment at various granularities or other linguistically meaningful units.
Again, we let $s_0$ denote the initial prompt and let $\tau=(s_0,a_0,\cdots, s_{T-1},a_{T-1},s_T)$ denote a trajectory generated under an action sequence $\boldsymbol{a}_{0:T-1}:=(a_0,a_1,\dots,a_{T-1})$. 
Given an initial prompt $s_0$, the search for an optimal sequence $\boldsymbol{a}^*(s_0)$ can be formulated by the following optimization problem
\begin{equation}
    \boldsymbol{a}^*(s_0):=\arg\max_{\boldsymbol{a}_{0:T-1}}\ \sum_{t=0}^{T-1} R(s_t,a_t).\label{eq:TO-main}
\end{equation}
Note that there is no need to take expectation in (\ref{eq:TO-main}) as the state transitions are deterministic given $\boldsymbol{a}_{0:T-1}$ in MDPs for text generation, as described in Section \ref{sec:background}.

\paragraph{Textual Model Predictive Control for Text Generation.} In general, direct optimization of (\ref{eq:TO-main}) requires searching over all possible action sequences of length $T$ and is computationally intractable. As a predictive planning method, MPC planner approximately solves (\ref{eq:TO-main}) by iteratively solving \textit{local} optimization problems~\citep{hansen2022temporal}, instead of globally optimizing the total reward in one pass. Specifically, MPC planner determines the action of each step $t$ by estimating the optimal subsequence $\boldsymbol{a}^*_{t:t+H}$ on a moving horizon $H$ (usually $H$ is smaller than $T$), given the state $s_t$, \ie
\begin{equation}
    \boldsymbol{a}^{\text{MPC}}(s_t):=\arg\max_{\boldsymbol{a}_{t:t+H-1}}\ 
    \sum_{i=t}^{t+H-1} R(s_t,a_t),\label{eq:MPC-main}
\end{equation}
and then select a subset of $\boldsymbol{a}^{\text{MPC}}(s_t)$, denoted by $\widetilde{\boldsymbol{a}}^{\text{MPC}}(s_t)$, for execution. In practice, $\widetilde{\boldsymbol{a}}^{\text{MPC}}(s_t)$ can be selected as the first $j$ contiguous actions ($1\leq j\leq H$) or as a set of non-contiguous actions~\citep{cagienard2007move}.
As a model-based approach, MPC solves (\ref{eq:MPC-main}) by employing (i) a learned predictive dynamics model and (ii) a proposal action distribution to jointly generate multiple $H$-step predictive rollouts $\{\tau^{(i)}_t\equiv(\boldsymbol{s}_{t:t+H-1}^{(i)},\boldsymbol{a}_{t:t+H-1}^{(i)})\}_{i=1}^{K}$ and obtain an approximate maximizer based on these $K$ rollouts. 
As a widely-used variant of MPC for continuous control, Model Predictive Path Integral (MPPI)~\citep{MPPI} determines an approximate maximizer by performing a soft, utility-weighted aggregated selection as $
{a}_t = \big({\sum_{i=1}^K \exp(\frac{1}{\lambda} \mathcal{J}(\tau_t^{(i)})) {a}_t^{(i)}}\big)/{\sum_{i=1}^K \exp(\frac{1}{\lambda} \mathcal{J}(\tau_t^{(i)}))}
$, where $\mathcal{J}(\tau)$ denotes the cumulative reward of a rollout $\tau$ and $\lambda > 0$ controls the exploration–exploitation trade-off. Compared to deterministic MPC that selects a single maximizer, MPPI yields smoother updates by aggregating multiple high-reward rollouts while still biasing toward higher $\mathcal{J}$.

Inspired by MPPI for continuous control, to better leverage MPC in text generation (inherently with discrete actions), we propose to define an \textit{aggregation function} that determines the action sequence by aggregating multiple textual rollouts based on the corresponding cumulative rewards, \ie
\begin{equation}
    \boldsymbol{a}^{\text{TMPC}}(s) \leftarrow \mathcal{G}\Big(\{\tau^{(i)}\}_{i=1}^{K},\{\mathcal{J}(\tau^{(i)})\}_{i=1}^{K};s\Big),
\end{equation}
where $\{\tau^{(i)}\}_{i=1}^{K}$ are rollouts starting from $s$. Then, TMPC can leverage a sequence of non-contiguous actions, denoted by $\widetilde{\boldsymbol{a}}^{\text{TMPC}}(s)$, to be selected for actual use in subgoal generation. The detailed construction of $\mathcal{G}$ will be specified in Section \ref{sec:general TMPC}.



Notably, TMPC enjoys two salient features that make it a particularly suitable method for test-time alignment of LLMs: (i) \textit{No additional model learning or fine-tuning needed}: Recall that MPC-like methods typically require a learned dynamics model and a proposal distribution. In TMPC, a dynamics model is already available since in text-generation MDPs, the transition from $s_t$ to $s_{t+1}$ is known and deterministic under an action $a_t$. Moreover, a pre-trained frozen LLM can naturally play the role of a good proposal distribution for generating candidate texts. Hence, TMPC does not require any fine-tuning or model learning. (ii) \textit{Addressing curse of horizon and curse of dimensionality}: TMPC addresses these two fundamental issues by iteratively solving local optimization problems. Compared to guided decoding and full-response iterative refinement, the design of TMPC can achieve a better balance between accurate credit assignment and the size of search space.

\begin{figure*}
  \centering
  \includegraphics[width=\linewidth]{TMPC.pdf}
  \caption{
    TMPC adapts the MPPI framework for test-time alignment by introducing two core principles. 
    \textbf{Hindsight Subgoal Identification:} After generating multiple rollouts, the planner's aggregation function $\mathcal{G}$ selects a subset of locally-optimal actions $\widetilde{\boldsymbol{a}}^{\text{TMPC}}$. This executed plan is retrospectively identified as a high-quality \textbf{subgoal} and stored in a buffer $\mathcal{B}$ if its utility meets a threshold $\alpha$. 
    \textbf{Subgoal-Conditioned Re-Generation:} New rollouts are generated by sampling from and composing subgoals in the buffer $\mathcal{B}$. This allows the planner to iteratively refine the full-horizon plan by building upon the best strategies discovered in previous iterations.
}
\label{fig:Plan2Align}
\end{figure*}

\subsection{Textual Model Predictive Control for General Temporal Progression}
\label{sec:general TMPC}
In this section, we extend the TMPC framework to text generation tasks with general temporal progression.
Inherited from the classic MPC, TMPC described in Section \ref{sec:TO} presumes that there already exists a basic unit as a discrete time step for planning. 
This requirement indeed holds for various tasks, such as viewing one output sentence as a step in machine translation and text summarization.
However, there also exist text generation tasks without natural boundaries, such as code generation. 
Despite this, we present a more general version of TMPC that can achieve approximate trajectory optimization in text generation, \textit{with and without natural boundaries}, by introducing \textit{subgoals}, which can serve as a basic unit for temporal progression. 
More specifically, subgoals provide directional guidance for the LLM’s generation, enabling efficient exploration toward the optimum.
TMPC can be substantiated via two core principles, as illustrated in Figure~\ref{fig:Plan2Align}.








\paragraph{Principle 1: Hindsight Subgoal Identification.}

To achieve higher-quality generation, we construct meaningful subgoals from continuous text by aggregating prior high-reward actions into a buffer $\mathcal{B}$. This identification occurs \textbf{after} rollouts are evaluated, hence hindsight, the planner discovers what constitutes a successful step based on empirical outcomes. The update rule of the buffer is as follows:

\begin{equation}
\mathcal{B} \leftarrow 
    \begin{cases} 
        \mathcal{B} \cup \widetilde{\boldsymbol{a}}_t^{\text{TMPC}}(s), & \text{if } |\mathcal{B}| < \text{capacity}, \\[6pt]
        \mathcal{B} \setminus \{a \in \mathcal{B} \mid R(s,a) < R(s,a') \} \cup \{a'\}, & \text{otherwise, for each } a' \in \widetilde{\boldsymbol{a}}_t^{\text{TMPC}}(s).
    \end{cases}
\end{equation}

\paragraph{Principle 2: Subgoal-Conditioned Aggregation Function for Re-Generation.}

In TMPC, the non-contiguous actions are generated from the following aggregation function:
\begin{align}
     \widetilde{\boldsymbol{a}}^{\text{TMPC}}_t(s) \leftarrow  \mathcal{G}\Big(\{\tau_t^{(i)}\}_{i=1}^{K}, R(\cdot) \mid s, \mathcal{B}\Big) :=  \left\{ a \mid R(s,a) \geq \alpha \text{ and } a \in \{\tau_t^{(i)}\}_{i=1}^{K} \right\}, \label{eq:sub goal compute}
\end{align}
where $\{\tau_t^{(i)}\}_{i=1}^{K}$ are the rollouts generated from subgoal-conditioned LLM $\pi(s,\mathcal{B})$.  
$\widetilde{\boldsymbol{a}}_t^{\text{TMPC}}(s)$ implicitly favors higher-reward outcomes by exploiting subgoals that serve as local optimizers over planning iterations, making it a validated and locally optimal action sequence with high utility.


This principle describes how TMPC leverages identified subgoals to refine the entire trajectory over multiple iterations. A single pass of optimization may yield a suboptimal solution. TMPC overcomes this by performing planning iteratively.
In the subgoal identification step, the planner populates the subgoal buffer $\mathcal{B}$ using the Hindsight Subgoal Identification described above. The re-generation step constructs new rollouts by explicitly leveraging the high-reward goals accumulated in $\mathcal{B}$ as conditioning signals. Rather than exploring from a generic proposal distribution, the planner is encouraged to generate new candidate trajectories by composing and extending the high-quality subgoals from the buffer. 
The aggregation function $\mathcal{G}$ thus plays a crucial role: it not only selects high-reward action subset $\widetilde{\boldsymbol{a}}^{\text{TMPC}}_t$ for the current iteration but also leverages the subgoal buffer $\mathcal{B}$ to inform the generation of rollouts for the next iteration. This iterative process allows TMPC to escape poor local optima and progressively construct a globally high-utility response by combining the best building blocks (subgoals) discovered across all iterations.


\section{Experiments}
\label{sec:experiments}

We evaluate TMPC on three tasks with different structural properties to ensure its generality: 
\textbf{(1) Paragraph-Level Machine Translation} represents a a task \textit{with natural boundaries}. The generated translation can be precisely aligned with the source text, allowing for sentence-level segments that are structurally anchored and easy to evaluate.
\textbf{(2) Long-Form Response Generation} represents a task \textit{without natural boundaries}. Without a source for direct alignment, responses are segmented by content into coherent chunks (\eg groups of sentences), each preserving semantic integrity.
\textbf{(3) Program Synthesis} challenges conventional segmentation, representing a task where structural boundaries (\eg Abstract Syntax Tree nodes) are semantically too fragmented for effective planning. Our framework addresses this by defining a segment abstractly through a functional milestone: the successful resolution of a single unit test.

\subsection{Preference Dataset and Reward Model}
\noindent\textbf{Paragraph-Level MT Dataset.} 
To construct a suitable preference dataset for long-text MT, we use the WMT'24 Discourse-Level Literary Translation benchmark \citep{wang2024findings} for our experiments. 
The available language pairs include: Chinese {\textrightarrow} English, Chinese {\textrightarrow} German, and Chinese {\textrightarrow} Russian.
To fit within LLM context windows, each instance is segmented into up to 1024 tokens using GPT-4’s tokenizer, ensuring paragraph-level MT remains within model limits.

The preference dataset is derived from the training set of the dataset. Each instance is segmented into paragraphs of up to 1,024 tokens. From each translation direction, we sample 2,000 paragraphs, resulting in a total of 6,000 paragraphs for constructing the preference dataset. Translation outputs are generated using LLaMA-3.1-8B-Instruct, Gemma-2-9B, and GPT-4o. The translations are then evaluated with MetricX-24-XL \citep{juraska-etal-2024-metricx} under the reference-free evaluation mode, where no reference translation is supplied as input. Following the procedure in CPO \cite{xu2024contrastive}, we assign the translation with the highest score as the \texttt{chosen} response, the one with the lowest score as the \texttt{rejected} response, and discard the middle-scoring translation. The resulting reward model achieves 88.53\% validation accuracy. Further details on the formation of preference data can be found in Appendix~\ref{sec:preference}, and detail of training can be found in Appendix~\ref{sec:training}.

\noindent\textbf{Long-Form Response Dataset.} 
We use the \texttt{Dahoas/full-hh-rlhf}\footnote{\url{https://huggingface.co/datasets/Dahoas/full-hh-rlhf}} dataset, which is widely adopted for LLM alignment. This dataset is designed to improve AI assistant behavior in terms of helpfulness and harmlessness. Each sample consists of a prompt and two responses, with one labeled as preferred based on human judgments.
Since the response lengths in the dataset vary significantly, we select samples based on the length of the \texttt{chosen} responses. Specifically, we construct the training set using the top 6K samples with the longest \texttt{chosen} responses from training set, and using the top 1024 longest \texttt{chosen} responses from the testing set to construct test set. We use the 6k size training set to train a reward model, which achieves a validation accuracy of 83.78\%. 

\noindent\textbf{Program Synthesis Dataset.}
We evaluate performance on the official testing set of the Mostly Basic Python Programming (MBPP) dataset~\citep{mbpp}, which comprises 500 problems (Task IDs 11-510). As discussed, code generation offers a direct reward signal. The resulting pass rate serves as the direct reward signal, eliminating the need for a separate reward model.

\subsection{Evaluation Metrics}
\noindent\textbf{Paragraph-Level MT.}
We use SEGALE~\citep{wang2025segale}, a framework that extends existing metrics to long-text translation. Following CPO~\citep{xu2024contrastive}, we apply COMET\footnote{\texttt{Unbabel/wmt22-comet-da}} within the SEGALE framework, thereby extending COMET to the paragraph level. 
To better capture contextual quality, rather than feeding only source, translation, and reference sentences into COMET, we follow \citet{vernikos-etal-2022-embarrassingly} and incorporate three concatenated sentences as inputs. We refer to this context-augmented version as \textbf{$\text{SEGALE}_\text{comet}$}.
SEGALE further reports the \textbf{Null Alignment (NA) Ratio}, the proportion of source or translation sentences that fail to align, often due to over- or under-translation. 

\noindent\textbf{Long-Form Responses.}
We evaluate response quality using two complementary metrics:  
\textbf{Average Reward} measures the mean score assigned by the reward model. This reflects the degree of alignment with helpfulness and harmlessness preferences. 
We introduce this metric to directly test whether TMPC achieves stronger alignment when the reward model and evaluation are consistent.
To avoid the potential for “cheating” in reward-based scoring, we also report \textbf{Win Rate}, which captures the proportion of pairwise comparisons in which a model’s response is preferred over a reference response by GPT-4~\citep{openai2024gpt4technicalreport}. Following the ARGS evaluation protocol~\citep{khanov2024args}, GPT-4 is prompted to assess overall response quality, considering helpfulness, harmlessness, relevance, accuracy, depth, creativity, and detail. The full evaluation prompt is provided in Appendix~\ref{sec:gptevalprompt}.  

\noindent\textbf{Program Synthesis.}
Following standard practice, we directly report the \textbf{Pass Rate}, defined as the proportion of problems for which all associated test cases are passed.

\subsection{Baselines}
We evaluate all training-time alignment methods on LLaMA-3.1-8B-Instruct and also adopt it as the backbone for all test-time alignment methods, including TMPC. 
Implementation details of TMPC, including parameters and prompt design, are provided in Appendix~\ref{sec:imple:TMPC}.  

\noindent\textbf{Test-Time Alignment Methods.}
We compare TMPC against the following representative approaches. 
(1) \textbf{ARGS}~\citep{khanov2024args}, a token-level decoding method that incorporates reward model guidance during inference. 
(2) \textbf{RAIN}~\citep{li2024rain}, which leverages tree-structured self-evaluation without relying on an external reward model. 
(3) \textbf{RE-Control}~\citep{re-control}, which modifies internal representations by training a value function on hidden states with the Bellman equation and applying gradient-based optimization to align preferences. 
(4) \textbf{GenARM}~\citep{GenARM}, an approach that trains an autoregressive reward model to assign token-level rewards conditioned on past tokens, and combines these reward scores with next-token probabilities during inference. 
(5) \textbf{TPO}~\cite{tpo}, which translates reward signals into textual critiques and uses an LLM to provide feedback for iterative refinement.
(6) \textbf{Best-of-N Sampling}, a widely adopted baseline that generates multiple candidates and selects the highest-scoring one. 

To ensure fair comparison, ARGS and RE-Control are equipped with the same reward model as TMPC. RAIN requires neither a reward model nor additional training data. 
GenARM trains its own autoregressive reward model using the same training data employed for TMPC's reward model. 
For TPO, we set the number of iterations to 4 to ensure it generates no fewer responses than TMPC, although this involves more LLM calls for textual losses and gradients. Further implementation details for all baselines, including a breakdown of TPO's LLM calls, are provided in Appendix~\ref{sec:test-time imp}.

\noindent\textbf{Training-Time Alignment Methods.}
We further compare TMPC with training-time alignment methods. We include supervised fine-tuning (SFT) on the same preference dataset, which often serves as a strong baseline in translation. In addition, we evaluate SimPO~\citep{meng2024simpo} and DPO~\citep{rafailov2023direct}, which represent recent and mainstream approaches to preference-based training-time alignment, respectively. Details of training procedures are reported in Appendix~\ref{sec:training}.  

\noindent\textbf{Task-Specific Settings.}
For paragraph-level MT, we include two high-performance models for additional context: \textbf{GPT-4o}, which serves as a strong upper bound despite not being specialized for translation~\citep{shahriar2024putting}, and \textbf{Qwen-2.5-14B}, a competitive open-source alternative for Chinese language tasks. For program synthesis, our comparison focuses on Best-of-N sampling and TPO. Token-level guided decoding methods are excluded as functional correctness is a holistic property of the entire code sequence, making them ill-suited for this task.

\subsection{Quantitative Results}
\noindent\textbf{Results on Paragraph-Level MT.} 
As shown in Table~\ref{tab:main}, TMPC consistently outperforms all test-time alignment baselines. It notably surpasses a strong Best-of-60 baseline with a fraction of the computational budget, underscoring the efficiency of predictive planning over naive sampling. For the zh→en direction, TMPC's performance even exceeds GPT-4o, highlighting its effectiveness on complex alignment tasks. TMPC's success stems from mitigating the failure modes of other paradigms. For instance, TPO exhibits inconsistent performance; while competitive in zh→ru, it is prone to factual inconsistencies in zh→en and zh→de, reflected in high NA Ratios.  Similarly, while RE-Control is more stable than myopic methods like ARGS and RAIN, it still underperforms and lacks a strategic refinement mechanism. TMPC inherits the stability of response-level refinement while avoiding the compounding errors of token-level guidance, striking a more effective balance.




\begin{table*}[t]
\centering
\begin{center}
\resizebox{\textwidth}{!}{%
\begin{tabular}{lccccccccc}
\toprule
\hline
\addlinespace
 & & \multicolumn{2}{c}{zh $\rightarrow$ en} & \multicolumn{2}{c}{zh $\rightarrow$ ru}  & \multicolumn{2}{c}{zh $\rightarrow$ de} \\
\cmidrule(lr){3-4} \cmidrule(lr){5-6} \cmidrule(lr){7-8} 
\multirow{-2}{*}{Methods}&\multirow{-2}{*}{\small Test-Time} & {\small  $\text{SEGALE}_\text{comet}$~$\uparrow$} & {\small NA Ratio~$\downarrow$} &  {\small $\text{SEGALE}_\text{comet}$~$\uparrow$} & {\small NA Ratio~$\downarrow$} & {\small $\text{SEGALE}_\text{comet}$~$\uparrow$}  & {\small NA Ratio~$\downarrow$}\\
\addlinespace
\midrule
\hline
\addlinespace
GPT-4o$_{\text{~2024-08-06}}$ & - & 94.58 & 0.10 & 93.74 & 0.00 & 94.54 & 0.00 \\
Qwen-2.5 (14B) & - & 94.43 & 0.18 & 90.47 & 3.08 & 92.98 & 1.24 \\
Llama-3.1 (8B) & $\times$& 84.36 & 10.47 & 86.28  & 4.19 & 88.97 & 4.43 \\
\addlinespace
\hline
\addlinespace
\cellcolor[HTML]{EFEFEF}Llama-3.1$_{\text{SFT}}$& \cellcolor[HTML]{EFEFEF}$\times$ & \cellcolor[HTML]{EFEFEF}93.54 & \cellcolor[HTML]{EFEFEF}0.34 & \cellcolor[HTML]{EFEFEF}89.11 & \cellcolor[HTML]{EFEFEF}1.92 & \cellcolor[HTML]{EFEFEF}93.47 & \cellcolor[HTML]{EFEFEF}0.19 \\
\cellcolor[HTML]{EFEFEF}Llama-3.1$_{\text{SimPO}}$& \cellcolor[HTML]{EFEFEF}$\times$ &\cellcolor[HTML]{EFEFEF}91.74 &\cellcolor[HTML]{EFEFEF}1.66 & \cellcolor[HTML]{EFEFEF}84.56 & \cellcolor[HTML]{EFEFEF}2.53  & \cellcolor[HTML]{EFEFEF}93.40 & \cellcolor[HTML]{EFEFEF}0.00\\
\cellcolor[HTML]{EFEFEF}Llama-3.1$_{\text{DPO}}$& \cellcolor[HTML]{EFEFEF}$\times$ & \cellcolor[HTML]{EFEFEF}90.23 & \cellcolor[HTML]{EFEFEF}1.33 & \cellcolor[HTML]{EFEFEF}82.15 & \cellcolor[HTML]{EFEFEF}6.62 & \cellcolor[HTML]{EFEFEF}93.48 & \cellcolor[HTML]{EFEFEF}0.00 \\
\addlinespace
\hline
\addlinespace
Llama-3.1$_{\text{ARGS}}$& \checkmark & 63.99 & 31.53 & 43.03 & 32.96 & 51.97 & 40.01\\
Llama-3.1$_{\text{RAIN}}$& \checkmark & 58.52 & 37.18 & 66.29 & 27.79 & 67.43 & 27.15 \\
Llama-3.1$_{\text{RE-Control}}$& \checkmark & 86.39 & 7.06 & 84.97 & 5.83 & 87.16 & \underline{5.96} \\
Llama-3.1$_{\text{GenARM}}$& \checkmark & 61.18 & 34.73 & 55.67 & 39.52 & 60.96 & 34.58 \\
Llama-3.1$_{\text{TPO}}$& \checkmark & 88.81 & 5.63 & \textbf{92.63} & \textbf{0.67} & \underline{87.67} & 6.79 \\
Llama-3.1$_{\text{Best-of-60}}$& \checkmark & \underline{90.97} & \underline{3.58} & 84.86 & 3.89 & 82.74 & 10.78\\
\cellcolor[HTML]{FFFFC7}Llama-3.1$_{\text{TMPC}}$ &\cellcolor[HTML]{FFFFC7} \checkmark & \cellcolor[HTML]{FFFFC7}\textbf{94.62} & \cellcolor[HTML]{FFFFC7}\textbf{0.00}  & \cellcolor[HTML]{FFFFC7}\underline{91.53} & \cellcolor[HTML]{FFFFC7}\underline{1.19} & \cellcolor[HTML]{FFFFC7}\textbf{91.73} & \cellcolor[HTML]{FFFFC7}\textbf{2.40} \\
\addlinespace
\bottomrule
\hline
\end{tabular}
}
\end{center}
\caption{Results on the WMT'24 literary translation shared task (\texttt{zh}$\rightarrow$\texttt{xx} directions). Results are grouped into SoTA and base models, training-time alignment methods, and test-time alignment methods. For test-time methods, the best-performing results are \textbf{bold}, and the second-best are \underline{\textbf{underlined}}.  Proposed methods are \colorbox{lightyellow}{highlighted}.}
\label{tab:main}
\end{table*}


\begin{figure}[h]
\centering
\begin{subfigure}[t]{0.58\textwidth}
\centering
\includegraphics[width=\linewidth]{figures/avg_reward_methods.pdf}
\end{subfigure}\hfill
\begin{subfigure}[t]{0.38\textwidth}
\centering
\includegraphics[width=\linewidth]{figures/win_rate_short.png}
\end{subfigure}
\caption{Results on the long-form responses.
\textbf{Left}: Average reward across the base model, training-time baselines, and test-time alignment methods. 
\textbf{Right}: GPT-4 win rate of TMPC against DPO and Best-of-20.
All methods use LLaMA-3.1-8B-Instruct as the backbone for fair comparison.}
\label{fig:win_rate}
\end{figure}

\noindent\textbf{Results on Long-Form Responses.} 
We present the results in Figure~\ref{fig:win_rate}. 
TMPC outperforms the strongest training-time (DPO) and test-time (Best-of-20) baselines in head-to-head comparisons judged by GPT-4. The efficiency of TMPC is particularly notable: TMPC requires only 3 iterations with 3 rollouts each, in addition to the initial LLM output, totaling 10 generations. In contrast, Best-of-20 produces twice as many outputs but still underperforms, showing that its advantage stems from TMPC rather than sheer sampling volume. 
Furthermore, TMPC provides a more stable alignment path than other test-time paradigms\footnote{TPO results are reported at iteration=2 in HH-RLHF dataset, as iteration=4 led to out-of-memory errors.}. TMPC bypasses fragile textual critiques and mitigates error accumulation by iteratively planning from a buffer of validated subgoals.



\begin{wrapfigure}{r}{0.35\textwidth} %
    \centering
    \includegraphics[width=\linewidth]{figures/passrate_methods.pdf}
    \caption{The pass rates on MBPP. }
    \label{fig:code_gen_results}
\end{wrapfigure}


\noindent\textbf{Results on Program Synthesis.} 
As shown in Figure~\ref{fig:code_gen_results}, TMPC achieves a 61\% pass rate, outperforming all baselines. This result highlights the limitations of unstructured approaches. Best-of-N sampling, even with a large budget ($N=35$), is constrained by the model's initial capabilities and relies on sampling chance. 
TPO shows only marginal gains with more iterations, reaching a pass rate of just 48\% after 4 iterations.
In contrast, TMPC systematically explores solution pathways by building upon partially correctness. 
Instead of merely hoping for a correct answer, TMPC maximizes the possibility of constructing one, allowing it to completely solve problems.

\begin{figure}[h]
\centering
    \begin{subfigure}[t]{0.25\textwidth}
        \centering
        \includegraphics[width=\linewidth]{figures/hyper_iter3.pdf}
        \caption{Hyperparameter effect}
        \label{fig:longform_grouped}
    \end{subfigure}
    \hfill
    \begin{subfigure}[t]{0.25\textwidth}
        \centering
        \includegraphics[width=\linewidth]{figures/rm_iter3.pdf}
        \caption{Reward model impact}
        \label{fig:longform_grouped_2}
    \end{subfigure}
    \hfill
    \begin{subfigure}[t]{0.45\textwidth}
        \centering
        \includegraphics[width=\linewidth]{figures/translation_grouped_iclr.pdf}
        \caption{Translation zh$\rightarrow$en, $\text{SEGALE}_\text{comet}$.}
        \label{fig:translation_grouped}
    \end{subfigure}
\caption{
    Robustness and sensitivity analysis of TMPC. 
    \textbf{(a)} Robustness to hyperparameter choices, with performance varying by less than 0.1 points across different buffer and segment sizes. 
    \textbf{(b)} Robustness to imperfections in the reward signal, including both injected noise and lower accuracy. 
    \textbf{(c)} SEGALE\textsubscript{comet} scores across iterations on zh→en translation. The standard TMPC steadily improves with more iterations, while a degraded version mimicking naive iterative refinement stagnates.
}
\label{fig:iter_trends}
\end{figure}

\subsection{Robustness and Sensitivity Analysis}
\label{sec:robustness}
Figure~\ref{fig:iter_trends} illustrates TMPC's robustness and sensitivity on long-form responses (full numerical results are in Appendix~\ref{sec:robustness}). As shown in Figure~\ref{fig:longform_grouped}, the framework is insensitive to its core hyperparameter choices; variations in buffer and segment size alter the average reward by less than 0.1 points, with performance consistently remaining superior to other test-time alignment methods. Figure~\ref{fig:longform_grouped_2} further tests the framework's robustness to reward model quality. Using a weaker reward model has a limited negative impact despite disturbing the optimization direction, while injected reward noise has a much smaller effect. We employ GRM~\citep{grm} as the weaker RM, using the \texttt{Ray2333/GRM\_Llama3.1\_8B\_rewardmodel-ft} checkpoint, which achieves 77.54\% validation accuracy. This resilience to noise stems from TMPC's subgoal buffer, which progressively filters out low-quality subgoals.
For paragraph-level MT, we analyze the zh$\rightarrow$en direction to reduce confounds from the base model’s familiarity with specific languages. Figure~\ref{fig:translation_grouped} reports iteration-wise performance to illustrate the trend of improvement over time 
The results show that TMPC performance steadily improves up to three iteration, after which extra iterations lead to a slight decline. In contrast, reducing TMPC to naive iterative refinement (\texttt{buf=1}, \texttt{seg=1}) yields no initial gains and fails to improve with more iterations, highlighting the importance of TMPC’s two principles.


\section{Conclusion}
We introduced TMPC, a test-time predictive planning framework for preference alignment. Under the sequential decision-making view, existing methods suffer from two fundamental limitations: guided decoding operates at the token level and faces the {curse of horizon}, while iterative refinement operates at the response level and suffers from the {curse of dimensionality}. TMPC strikes a balance by identifying locally-optimal trajectory segments as subgoals in hindsight, and then leveraging a buffer of these subgoals to iteratively refine the full-horizon plan. This design mitigates both challenges and enables consistent improvements in long-form alignment without modifying the model parameters.


\section*{Ethics Statement}
This work introduces Textual Model Predictive Control (TMPC), a general framework for the test-time alignment of large language models. Our primary goal is to develop more stable and efficient methods for aligning models with beneficial human preferences, such as helpfulness and harmlessness. All experiments were conducted on publicly available and widely used academic benchmarks (HH-RLHF, WMT'24, and MBPP), and no new data involving human subjects was collected.

The primary ethical consideration of our work is that the alignment outcome is determined by the provided reward signal. While we have used it for positive alignment, a malicious or biased reward signal could steer a model toward generating harmful, unfair, or toxic content. TMPC, like other alignment techniques, could potentially amplify biases present in the preference data used to train the reward model. We therefore stress the importance of careful design, auditing, and red-teaming of reward models before deploying systems using this technology in sensitive, real-world applications. We believe that by providing a more transparent and controllable test-time alignment mechanism, our work can contribute positively to the development of safer AI systems.

\section*{Reproducibility Statement}
We are committed to ensuring the reproducibility of our research. Our implementation of TMPC, along with all experimental scripts, will be made publicly available in a permissively licensed open-source repository upon publication. 

The core methodology is described in Section~\ref{sec:methodology}, with a detailed, task-agnostic algorithm provided in Algorithm~\ref{alg:TMPC_unified}. All datasets used in our experiments---HH-RLHF, WMT'24, and MBPP---are public benchmarks, with details on their specific versions and preprocessing steps provided in Section~\ref{sec:experiments} and Appendix~\ref{sec:baselines_imp}. All hyperparameters, prompt templates, and task-specific implementation details necessary to replicate our results for long-form response generation, machine translation, and programmatic synthesis are documented in Appendix~\ref{sec:imple:TMPC}. We believe these resources provide a clear and sufficient basis for the community to reproduce and build upon our findings.


\bibliography{main}
\bibliographystyle{iclr2026_conference}

\appendix
% \section{Appendix}


\section{Action-Level Contrastive Reward}\label{action-level}

We made the distinction between action-level variables and token-level variables: action-level (or step-level) variables are those that aggregate over all tokens in a reasoning step, and is typically utilized by the reasoning algorithm directly; token-level variables, by contrast, operates in a more microscopic and low-level environment, such as speculative decoding.

We found that the traditional contrastive decoding using the difference in logits, when aggregated over the sequence gives a unstable reward signal compared to JS divergence. We suspected this is due to the unbounded nature of logit difference, and the potential failure modes associated with it that needs extra care and more hyperparameter tuning.



\section{More Related Work}
\label{related_of_decoding}
\paragraph{Large Language Models Multi-Step Reasoning}
Deepseek Prover~\citep{Xin2024DeepSeekProverAT, deepprover1.5} relied on Lean4 as an external verification tool to provide dense reward signals in the RL stage. ReST-MCTS$^*$~\citep{ReST-MCTS} employed self-training to collect high-quality reasoning trajectories for iteratively improving the value model. AlphaLLM~\citep{towardself} used critic models initialized from the policy model as the MCTS reward model. rStar~\citep{mutualReasoning} utilized mutual consistency of SLMs and an additional math-specific action space. \cite{xu2023traingainunleashmathematical} proposed reconstructing fine-tuned LLMs into residual-based energy models to guide MCTS.

\paragraph{Speculative Decoding}

Speculative decoding was first introduced in \citet{google_sd}, as a method to accelerate sampling from large autoregressive models by computing multiple tokens in parallel without retraining or changing the model structure. It enhances computational efficiency, especially in large-scale generation tasks, by recognizing that hard language-modeling tasks often include easier subtasks that can be approximated well by more efficient models. Similarly, DeepMind introduced speculative sampling~\citep{dm_sd}, which expands on this idea by generating a short draft sequence using a faster draft model and then scoring this draft with a larger target model.

\paragraph{Contrastive Decoding}

Contrastive decoding, as proposed by \citet{contrastivedecoding}, is a simple, computationally light, and training-free method for text generation that can enhancethe quality and quantity by identifying strings that highlight potential differences between strong models and weak models. In this context, the weak models typically employ conventional greedy decoding techniques such as basic sampling methods, while the strong models are often well-trained large language models. This approach has demonstrated notable performance improvements in various inference tasks, including arithmetic reasoning and multiple-choice ranking tasks, thereby increasing the accuracy of language models. According to experiments conducted by \cite{contrastiveandreasoning}, applying contrastive decoding across various tasks has proven effective in enhancing the reasoning capabilities of LLMs.


\section{Reward Functions Correlation}
\label{heatmap}

\begin{figure}[H]
    \centering
    \vspace{-2mm}
    \includegraphics[width=0.45\textwidth]{fig/heatmap.png}
    \vspace{-3mm}
    \caption{Reward Functions Correlation Heatmap.}
    \label{fig:heatmap}
\end{figure}

It can be seen from Figure \ref{fig:heatmap} that the correlations between the three reward functions are relatively low, absolute values all below 0.15. These low correlations of reward functions make them ideal for Multi-RM method.





\section{Algorithm Details of \texorpdfstring{\ours}{SC-MCTS*}}
\label{details_of_sc_mcts}
The pseudocode inside MCTS reasoning of SC-MCTS$^*$ is shown in Algorithm~\ref{alg:reason}, based on \citet{ReST-MCTS}. The complete version of \ours is: first sample a subset of problems to obtain the prior data for reward values (Algorithm~\ref{alg:reward-construct}), then use it and two SLMs, one for providing contrastive reward signals, another for speculative decoding speedup, to perform MCTS reasoning. The changes of SC-MCTS$^*$ compared to previous works are highlighted in {\color{teal} teal}. 

\newpage

\begin{algorithm*}[htbp]
\centering
\caption{SC-MCTS$^*$, reasoning}
\label{alg:reason}
\begin{algorithmic}[1]
\Require expert LLM $\pi_{\text{e}}$, amatuer SLM $\pi_{\text{a}}$, speculative SLM $\pi_{\text{s}}$, problem $q$, reward model $R$, reward factor statistics $\gS$, max iterations $T$, threshold $l$, branch $b$, rollout steps $m$, roll branch $d$, weight parameter $\alpha$, exploration constant $C$
\State $T_q \gets$ Initialize-tree$(q)$
\For {$i = 1 \ldots T$}
	\State $n \gets$ Root$(T_q)$
	\While {$n$ is not leaf node} \Comment{Node selection}
		\State {\color{teal} $n \gets$ $\argmax_{n'\in \text{children}(n)}(v_{n'}+C\sqrt{\frac{\ln{N_n}}{N_{n'}}})$ \Comment{Select child node based on UCT}}
	\EndWhile
	\If {$v_n \geq l$}
		\textbf{break} \Comment{Output solution}
	\EndIf
	\If {$n$ is not End of Inference}
		\For {$j = 1 \ldots b$} \Comment{Thought expansion}
			\State $n_j \gets$ Get-new-child$(A_n, q, \pi_{\text{e}})$ \Comment{Expand based on previous steps}
			\State {\color{teal} $v_{n_j}, \gS \gets$ $R(A_{n_j}, q, \pi_{\text{e}}, \pi_{\text{a}}, \gS)$ \Comment{Evaluate contrastive reward and update reward factor statistics}}
		\EndFor
		\State $n' \gets$ $\argmax_{n'\in \text{children}(n)}(v_{n'})$
		\State $v_{\max} \gets$ 0
		\For {$k = 1 \ldots m$} \Comment{Greedy MC rollout}
			\State {\color{teal} $A, v_{\max} \gets$ Get-next-step-with-best-value$(A, q, \pi_{\text{e}}, \pi_{\text{s}}, d)$ \Comment{Sample new children using speculative decoding and record the best observed value}}
		\EndFor
		\State $v_{n'} \gets$ $\alpha v_{n'}+(1-\alpha)v_{\max}$
		\State $N_{n'} \gets$ $N_{n'}+1$ \Comment{Update value and visit count of the rollout node}
	\EndIf
	\State {\color{teal} Back-propagate$(n)$ \Comment{Update value of parent nodes (Equation~\ref{eq:backprop})}}
\EndFor
\State $n \gets$ Get-best-node$(T_q)$ \Comment{Fetch the node with the highest value in the search tree}
\Ensure $A_n$
\end{algorithmic}
\end{algorithm*}



Although we sampled a small portion of the dataset as prior data for reward values, distribution shift may still occur when normalizing reward values during reasoning. Therefore, we use the following algorithm to incrementally update the mean and standard deviation of the online reward distribution:

\begin{algorithm}[H]
\caption{Online incremental update of reward factor statistics}
\label{alg:3}
\begin{algorithmic}[1]
\Require reward factors $\mathcal{R} (= \{\text{JSD}, \text{LL}, \text{SE}\})$, 
         statistics $\{\mu_r^{(k)}, \sigma_r^{(k)}, n_r^{(k)}\}_{r \in \mathcal{R}, k \in \{1,\ldots,K\}}$,
         cluster assignment function $f$

    \For{$r \in \mathcal{R}$}
        \State $k^* \gets f(x)$ \Comment{Assign sample to cluster}
        \State $v_r \gets r(x)$ \Comment{Compute reward factor value}
        \State $n_r^{(k^*)} \gets n_r^{(k^*)} + 1$ \Comment{Update sample count}
        \State $\delta \gets v_r - \mu_r^{(k^*)}$ \Comment{Compute difference from mean}
        \State $\mu_r^{(k^*)} \gets \mu_r^{(k^*)} + \delta / n_r^{(k^*)}$ \Comment{Update mean}
        \State $M_2 \gets (n_r^{(k^*)}-1)(\sigma_r^{(k^*)})^2 + \delta(v_r - \mu_r^{(k^*)})$
        \State $\sigma_r^{(k^*)} \gets \sqrt{M_2 / n_r^{(k^*)}}$ \Comment{Update standard deviation}
    \EndFor
\Ensure updated statistics $\{\mu_r^{(k)}, \sigma_r^{(k)}, n_r^{(k)}\}_{r \in \mathcal{R}, k \in \{1,\ldots,K\}}$

\end{algorithmic}
\end{algorithm}



\newpage
\section{Experimental Settings}

\label{llm_details}

For reproducibility, you can download the checkpoints from the Huggingface repository below and use the hyperparameters below. We utilized 4-bit quantized checkpoints in all experiments, as they only result in around 2\% performance loss while providing several-fold reductions in memory usage and significantly improving inference speed~\citep{frantar2022gptq}. For better output formatting to capture a single step and convert it into an MCTS node, we used the LLM's completion mode so we set LLM to greedy sampling, and we don't have to set an additional system prompt, simply apply prompts in Appendix~\ref{sec:blocksworld_dataset}. Our experiments were all conducted on exllamav2 inference framework.

\vspace{6mm}
\subsection{Checkpoints}
\vspace{2mm}
\begin{table}[H]
\small
\label{tab:tab6}
\small

\centering
\renewcommand{\arraystretch}{1}

\setlength{\tabcolsep}{14pt}

\begin{tabular}%
{p{1cm}%
r% Right-align the Models column
p{8cm}%
}
\toprule

\multicolumn{1}{c}{\multirow{1}{*}{\textbf{Usage}}}
& \multicolumn{1}{c}{\multirow{1}{*}{\textbf{Models}}}
& \multicolumn{1}{c}{\multirow{1}{*}{\textbf{Links}}} \\
\midrule

\multirow{4}{*}{\textbf{ Expert}} 
& \textbf{Llama-3.1-405B} 
      & \tiny{\url{https://huggingface.co/hugging-quants/Meta-Llama-3.1-405B-Instruct-GPTQ-INT4}}      \\ 
& \textbf{Llama-3.1-70B} 
      & \tiny{\url{https://huggingface.co/hugging-quants/Meta-Llama-3.1-70B-Instruct-GPTQ-INT4}}      \\ 
& \textbf{Llama-3-70B} 
      & \tiny{\url{https://huggingface.co/TechxGenus/Meta-Llama-3-70B-Instruct-GPTQ}} \\
\midrule

\multirow{4}{*}{\textbf{Amateur}} 
& \textbf{Llama-3.1-8B} 
      & \tiny{\url{https://huggingface.co/hugging-quants/Meta-Llama-3.1-8B-Instruct-GPTQ-INT4}}      \\ 
& \textbf{Llama-3-8B} 
      & \tiny{\url{https://huggingface.co/astronomer/Llama-3-8B-Instruct-GPTQ-4-Bit}}      \\ 
& \textbf{Llama-3.2-1B} 
      & \tiny{\url{https://huggingface.co/meta-llama/Llama-3.2-1B}}      \\ 
\midrule


\multirow{2}{*}{\textbf{ OpenAI}} 
& \textbf{GPT-4o} 
      & \tiny{\url{https://platform.openai.com/docs/models/gpt-4o}}       \\
& \textbf{o1-mini} 
      & \tiny{\url{https://platform.openai.com/docs/models/o1}}      \\ 

\bottomrule
\end{tabular}
\vspace{2mm}
\caption{Checkpoints used in experiments and their links.} 
\end{table}

\vspace{6mm}
\subsection{Hyperparameters}
\vspace{4mm}
\begin{table}[htbp]
\centering
\begin{tabular}{|l|l|}
\hline
\textbf{Hyperparameter}                & \textbf{Value} \\ \hline
temperature                            & 1.0            \\ \hline
top-k                                  & 1.0            \\ \hline
top-p                                  & 1.0            \\ \hline
repetition\_penalty                    & 1.0            \\ \hline
max\_new\_tokens                       & 200            \\ \hline
max\_seq\_len                          & 32768          \\ \hline
MCTS EOS: \texttt{Llama-3 family}      & \texttt{"\textbackslash n[}" \\ \hline
CoT EOS: \texttt{Llama-3 family}       & \texttt{"\textbackslash n"}, \texttt{"\textless |eot\_id|\textgreater"} \\ \hline
\end{tabular}
\vspace{3mm}
\caption{LLM Hyperparameters and EOS tokens used in experiments.}
\label{tab:hyperparameters}
\end{table}






% \section{Experiment on SLMs}

% \begin{table}[H]
% \caption{We also conducted experiment on SLMs, due to...}
% \setlength{\tabcolsep}{10pt} 
% \vspace{10pt}
% \centering
% \resizebox{1.00\textwidth}{!}{
% \begin{tabular}{llcccccc}
% \toprule
%  & &\multicolumn{6}{c}{\textbf{Steps}} \\
% \cmidrule(r){3-8}
% Models & Method \\
% & & Step 2 & Step 4 & Step 6 & Step 8 & Step 10 & Step 12 \\
% \midrule
% \multirow{3}{*}{Llama-3.1-8B \ } 
%  & CoT & 0.3333 & 0.1071 & 0.0461 & 0.0066 & 0.0179 & 0.0000 \\ 
%  & RAP-MCTS & 0.6667 & 0.7381 & 0.6842 & 0.4702 & 0.1339 & 0.0652 \\
%  & \rowcolor{gray!20}\ours (Ours) & \rowcolor{gray!20}0.0 & \rowcolor{gray!20}0.0 & \rowcolor{gray!20}0.0 & \rowcolor{gray!20}0.0 & \rowcolor{gray!20}0.0 & \rowcolor{gray!20}0.0 \\
% \midrule
% \multirow{3}{*}{Llama-3-8B} 
%  & CoT & 0.2000 & 0.0595 & 0.0592 & 0.0000 & 0.0089 & 0.0000 \\ 
%  & RAP-MCTS & 0.6222 & 0.6667 & 0.6053 & 0.4040 & 0.1250 & 0.0000 \\
%  & \rowcolor{gray!20}\ours (Ours) & \rowcolor{gray!20}0.0 & \rowcolor{gray!20}0.0 & \rowcolor{gray!20}0.0 & \rowcolor{gray!20}0.0 & \rowcolor{gray!20}0.0 & \rowcolor{gray!20}0.0 \\
% \midrule
% \multirow{3}{*}{Llama-2-7B \ } 
%  & CoT & 0.3111 & 0.0595 & 0.0132 & 0.0000 & 0.0089 & 0.0000 \\ 
%  & RAP-MCTS & 0.5556 & 0.5952 & 0.4145 & 0.1987 & 0.0446 & 0.0222 \\
%  & \rowcolor{gray!20}\ours (Ours) & \rowcolor{gray!20}0.0 & \rowcolor{gray!20}0.0 & \rowcolor{gray!20}0.0 & \rowcolor{gray!20}0.0 & \rowcolor{gray!20}0.0 & \rowcolor{gray!20}0.0 \\
% \midrule[1pt]
% \addlinespace[3pt]
% \midrule[0.5pt]

% \end{tabular}
% }
% \label{table:1}
% \end{table}




\newpage
\section{Blocksworld Dataset}
\label{sec:blocksworld_dataset}

The Blocksworld dataset comprises 600 instances with varying block numbers and plan lengths. Simpler instances have 3-5 blocks, while more complex cases involve up to 25 blocks, introducing additional goals and obstacles. This setup covers a range of problem difficulties for evaluating planning algorithms. 

\subsection{Difficulty Settings}

According to settings of LLM Reasoners~\citep{llmreaonser}, we divide the original 600 instances of Blocksworld~\citep{b1} into two parts, Easy and Hard settings.

In the Easy Blocksworld setting, we use more friendly demonstration cases. If a problem requires a specific minimum number of steps to solve, we select other problems that require the same number of steps as demonstration cases in the context. For example, if a problem requires at least 4 steps to solve, we use other 4-step problems as demonstration examples. For each group of problems, we randomly select 10 cases to create a pool of demonstration cases, while the remaining cases form the test set (a total of 540 cases). During inference, we randomly sample 4-shot demonstration cases from this pool to construct the prompts.


In the Hard Blocksworld setting, we randomly select 10 cases from the entire dataset to create the demonstration pool. These selected cases are then excluded from the test set, leaving a total of 590 cases for testing. During inference, we randomly sample 4-shot demonstration cases from this global pool, without considering the minimum number of actions required for the test case. For example, if a problem requires at least 4 steps to solve, we may still use demonstration cases that require a different number of steps, such as 2 or 12, as there is no restriction based on the number of actions.


\begin{table}[h]\centering
\begin{minipage}{1.0\textwidth}
\centering
\begin{tcolorbox} 
\centering
\small
\begin{tabular}{p{0.8\textwidth}}\\
   
\textbf{domain\_intro:} \\   
   
\textbf{I am playing with a set of objects. Here are the actions I can do:} \\
pick up a block \\
unstack a block from on top of another block \\
put down a block \\
stack a block on top of another block \\ \\

\textbf{I have the following restrictions on my actions:}

To perform the Pick Up action, the block must be clear, on the table, and my hand must be empty. Once the Pick Up action is performed, I am holding the block, and my hand is no longer empty. \\ \\

To perform the Unstack action, the block must be clear, on top of another block, and my hand must be empty. Once the Unstack action is performed, I am holding the block, and my hand is no longer empty. \\ \\

To perform the Put Down action, I must be holding a block. Once the Put Down action is performed, the block is on the table, my hand is empty, and the block becomes clear. \\ \\

To perform the Stack action, I must be holding a block, and the block I want to stack it on must be clear. Once the Stack action is performed, the block is on top of another block, my hand is empty, and the block on top is no longer clear.

\end{tabular}
\end{tcolorbox}
%\vspace{-2mm}
\caption{Normal Blocksworld Task Setting}
\label{tab: Normal Blocksworld Setting}
\end{minipage}
\end{table}





% \begin{table}[h!]\centering
% \begin{minipage}{0.95\textwidth}
    
% \centering
% \begin{tcolorbox} 
%     \centering
   
%       \small
%     \begin{tabular}{p{0.95\textwidth}} \hline \\
%    \textbf{Mystery Blocksworld Setting} \\    
   
%    % AI that scores image description accuracy and detailedness.

%    \\ \midrule

%    % \textbf{domain_intro:} \\   
   
% \textbf{I am playing with a set of objects. Here are the actions I can do:}
% \begin{itemize}
%     \item Attack object
%     \item Feast object from another object
%     \item Succumb object
%     \item Overcome object from another object
% \end{itemize}

% \textbf{I have the following restrictions on my actions:}
% \begin{itemize}
%     \item \textbf{To perform Attack action, the following facts need to be true:} Province object, Planet object, Harmony.
%     \item \textbf{Once Attack action is performed the following facts will be true:} Pain object.
%     \item \textbf{Once Attack action is performed the following facts will be false:} Province object, Planet object, Harmony.
    
%     \item \textbf{To perform Succumb action, the following facts need to be true:} Pain object.
%     \item \textbf{Once Succumb action is performed the following facts will be true:} Province object, Planet object, Harmony.
%     \item \textbf{Once Succumb action is performed the following facts will be false:} Pain object.
    
%     \item \textbf{To perform Overcome action, the following needs to be true:} Province other object, Pain object.
%     \item \textbf{Once Overcome action is performed the following will be true:} Harmony, Province object, Object Craves other object.
%     \item \textbf{Once Overcome action is performed the following will be false:} Province other object, Pain object.
    
%     \item \textbf{To perform Feast action, the following needs to be true:} Object Craves other object, Province object, Harmony.
%     \item \textbf{Once Feast action is performed the following will be true:} Pain object, Province other object.
%     \item \textbf{Once Feast action is performed the following will be false:} Object Craves other object, Province object, Harmony.
% \end{itemize}



% \bottomrule
%     \end{tabular}
% \end{tcolorbox}
% %\vspace{-2mm}
% \caption{Mystery Blocksworld Setting}
%     \label{tab: Mystery Blocksworld Setting}
% \end{minipage}
% \end{table}

% \begin{table}[h!]\centering
% \begin{minipage}{0.95\textwidth}
    
% \centering
% \begin{tcolorbox} 
%     \centering
   
%       \small
%     \begin{tabular}{p{0.95\textwidth}} \hline \\
%    \textbf{Randomized Blocksworld Setting} \\    
   
%    % AI that scores image description accuracy and detailedness.

%    \\ \midrule

%    % \textbf{domain_intro:} \\   
   
% I am playing with a set of objects. Here are the actions I can do:

% \begin{itemize}
%     \item \textbf{1jpkithdyjmlikck} (on an object)
%     \item \textbf{xptxjrdkbi3pqsqr} (an object from another object)
%     \item \textbf{9big8ruzarkkquyu} (on an object)
%     \item \textbf{2ijg9q8swj2shjel} (an object from another object)
% \end{itemize}

% I have the following restrictions on my actions:
% \begin{itemize}
%     \item To perform \textbf{1jpkithdyjmlikck} action, the following facts need to be true: \textbf{aqcjuuehivl8auwt} object, \textbf{51nbwlachmfartjn} object, and \textbf{3covmuy4yrjthijd}.
%     \item Once \textbf{1jpkithdyjmlikck} action is performed, the following facts will be true: \textbf{gk5asm3f7u1fekpj} object.
%     \item Once \textbf{1jpkithdyjmlikck} action is performed, the following facts will be false: \textbf{aqcjuuehivl8auwt} object, \textbf{51nbwlachmfartjn} object, and \textbf{3covmuy4yrjthijd}.
    
%     \item To perform \textbf{9big8ruzarkkquyu} action, the following facts need to be true: \textbf{gk5asm3f7u1fekpj} object.
%     \item Once \textbf{9big8ruzarkkquyu} action is performed, the following facts will be true: \textbf{aqcjuuehivl8auwt} object, \textbf{51nbwlachmfartjn} object, and \textbf{3covmuy4yrjthijd}.
%     \item Once \textbf{9big8ruzarkkquyu} action is performed, the following facts will be false: \textbf{gk5asm3f7u1fekpj} object.
    
%     \item To perform \textbf{2ijg9q8swj2shjel} action, the following needs to be true: \textbf{aqcjuuehivl8auwt} other object and \textbf{gk5asm3f7u1fekpj} object.
%     \item Once \textbf{2ijg9q8swj2shjel} action is performed, the following will be true: \textbf{3covmuy4yrjthijd}, \textbf{aqcjuuehivl8auwt} object, and \textbf{Object 4DMF1cMTYXGSP94G} other object.
%     \item Once \textbf{2ijg9q8swj2shjel} action is performed, the following will be false: \textbf{aqcjuuehivl8auwt} other object and \textbf{gk5asm3f7u1fekpj} object.
    
%     \item To perform \textbf{xptxjrdkbi3pqsqr} action, the following needs to be true: \textbf{Object 4DMF1cMTYXGSP94G} other object, \textbf{aqcjuuehivl8auwt} object, and \textbf{3covmuy4yrjthijd}.
%     \item Once \textbf{xptxjrdkbi3pqsqr} action is performed, the following will be true: \textbf{gk5asm3f7u1fekpj} object and \textbf{aqcjuuehivl8auwt} other object.
%     \item Once \textbf{xptxjrdkbi3pqsqr} action is performed, the following will be false: \textbf{Object 4DMF1cMTYXGSP94G} other object, \textbf{aqcjuuehivl8auwt} object, and \textbf{3covmuy4yrjthijd}.
% \end{itemize}



% \bottomrule
%     \end{tabular}
% \end{tcolorbox}
% %\vspace{-2mm}
% \caption{Randomized Blocksworld Setting}
%     \label{tab: Randomized Blocksworld Setting}
% \end{minipage}
% \end{table}



\subsection{Prompts Settings of Easy Blocksworld}




\begin{table}[H]\centering
\begin{minipage}{0.95\textwidth}
%\vspace{0mm}    
\centering
\begin{tcolorbox} 
    \centering
   
     %\hspace{-4mm}
      \small
    \begin{tabular}{p{0.95\textwidth}} \hline \\
   % \textbf{Description:} \\    
   
   % AI that scores image description accuracy and detailedness.

   % \\ \midrule
\textbf{Input Instructions:} \\

I am playing with a set of blocks where I need to arrange the blocks into stacks. Here are the actions I can do:
\begin{enumerate}
    \item Pick up a block
    \item Unstack a block from on top of another block
    \item Put down a block
    \item Stack a block on top of another block
\end{enumerate}

I have the following restrictions on my actions:
\begin{enumerate}
    \item I can only pick up or unstack one block at a time.
    \item I can only pick up or unstack a block if my hand is empty.
    \item I can only pick up a block if the block is on the table and the block is clear. A block is clear if the block has no other blocks on top of it and if the block is not picked up.
    \item I can only unstack a block from on top of another block if the block I am unstacking was really on top of the other block.
    \item I can only unstack a block from on top of another block if the block I am unstacking is clear.
\end{enumerate}

Once I pick up or unstack a block, I am holding the block.

\begin{enumerate}
    \item I can only put down a block that I am holding.
    \item I can only stack a block on top of another block if I am holding the block being stacked.
    \item I can only stack a block on top of another block if the block onto which I am stacking the block is clear.
\end{enumerate}

Once I put down or stack a block, my hand becomes empty.



\\
% \\ \\
% \textbf{CoT Format:} \\ \\
\lbrack{}STATEMENT\rbrack{}\\
 As initial conditions I have that, the red block is clear, the hand is empty, the blue block is on top of the orange block, the red block is on the table, the orange block is on the table and the yellow block is on the table.\\
My goal is to have that the orange block is on top of the blue block. 
My plan is as follows:\\
\lbrack{}End Of STATEMENT\rbrack{} \\
\\
\lbrack{}PLAN\rbrack{} \\
unstack the blue block from on top of the orange block \\
put down the blue block \\
pick up the orange block \\
stack the orange block on top of the blue block \\
\lbrack{}PLAN END\rbrack{} \\
\\
\lbrack{}STATEMENT\rbrack{}\\
As initial conditions I have that, the red block is clear, the yellow block is clear, the hand is empty, the red block is on top of the blue block, the yellow block is on top of the orange block, the blue block is on the table and the orange block is on the table.\\
My goal is to have that the orange block is on top of the red block.
My plan is as follows:\\
\lbrack{}End Of STATEMENT\rbrack{} \\

\\
\textbf{Output format:}\\
\lbrack{}PLAN\rbrack{} \\
\textbf{[LLM Completion]}
\\ \lbrack{}PLAN\_END\rbrack{} \\

\bottomrule
    \end{tabular}
\end{tcolorbox}
%\vspace{-2mm}
\caption{The Prompt Settings for Easy Blocksworld}
    \label{tab:easy_Pormpt}
\end{minipage}
\end{table}








\subsection{Prompts Settings of Hard Blocksworld}


\begin{table}[H]\centering
\begin{minipage}{0.95\textwidth}
%\vspace{0mm}    
\centering
\begin{tcolorbox} 
    \centering
   
     %\hspace{-4mm}
      \small
    \begin{tabular}{p{0.95\textwidth}} \hline \\
   % \textbf{Description:} \\    
   
   % AI that scores image description accuracy and detailedness.

   % \\ \midrule

\textbf{Input Instructions:} \\   
   
I am playing with a set of blocks where I need to arrange the blocks into stacks. Here are the actions I can do:
\begin{enumerate}
    \item Pick up a block
    \item Unstack a block from on top of another block
    \item Put down a block
    \item Stack a block on top of another block
\end{enumerate}

I have the following restrictions on my actions:
\begin{enumerate}
    \item I can only pick up or unstack one block at a time.
    \item I can only pick up or unstack a block if my hand is empty.
    \item I can only pick up a block if the block is on the table and the block is clear. A block is clear if the block has no other blocks on top of it and if the block is not picked up.
    \item I can only unstack a block from on top of another block if the block I am unstacking was really on top of the other block.
    \item I can only unstack a block from on top of another block if the block I am unstacking is clear.
\end{enumerate}

Once I pick up or unstack a block, I am holding the block.

\begin{enumerate}
    \item I can only put down a block that I am holding.
    \item I can only stack a block on top of another block if I am holding the block being stacked.
    \item I can only stack a block on top of another block if the block onto which I am stacking the block is clear.
\end{enumerate}

Once I put down or stack a block, my hand becomes empty.

\\
% \\ \\
% \textbf{CoT Format:} \\ \\
\lbrack{}STATEMENT\rbrack{}\\
 As initial conditions I have that, the blue block is clear, the hand is empty, the blue block is on top of the red block, the red block is on the table, the orange block is on the table and the yellow block is on the table.\\
My goal is to have that the blue block is on top of the orange block. 
My plan is as follows:\\
\lbrack{}End Of STATEMENT\rbrack{} \\
\\
\lbrack{}PLAN\rbrack{} \\
unstack the blue block from on top of the red block \\
stack the blue block on top of the orange block \\
\lbrack{}PLAN END\rbrack{} \\
\\
\lbrack{}STATEMENT\rbrack{}\\
As initial conditions I have that, the red block is clear, the yellow block is clear, the hand is empty, the red block is on top of the blue block, the yellow block is on top of the orange block, the blue block is on the table and the orange block is on the table.\\
My goal is to have that the orange block is on top of the red block.
My plan is as follows:\\
\lbrack{}End Of STATEMENT\rbrack{} \\

\\
\textbf{Output format:}\\
\lbrack{}PLAN\rbrack{} \\
\textbf{[LLM Completion]}
\\ \lbrack{}PLAN\_END\rbrack{} \\

\bottomrule
    \end{tabular}
\end{tcolorbox}
%\vspace{-2mm}
\caption{The Prompt Settings for Hard Blocksworld}
    \label{tab:prompt_evaluation}
\end{minipage}
\end{table}





% \section{OpenAI API Data}
% \label{openai_api}
% \begin{table}[H]

% \centering
% % \small
% \setlength{\tabcolsep}{15pt}
% \begin{tabular}{@{}cS[table-format=1.2]S[table-format=1.2]S[table-format=1.2]@{}}
% \toprule
% \textbf{Difficulty} & \textbf{Model} & \textbf{USD per instance} & \textbf{Total Experiment Cost (USD)} \\ 
% \midrule
% \multirow{2}{*}{\textbf{\ \ Easy (0-shot)}} & GPT-4o & \$0.0032 & \$1.73 \\
%                                         & o1-mini & \$0.0136 & \$7.34 \\ 
% \midrule
% \multirow{2}{*}{\textbf{\ \ Easy (4-shot)}} & GPT-4o & \$0.0062 & \$3.35 \\
%                                         & o1-mini & \$0.0171 & \$9.23 \\ 
% \midrule
% \multirow{2}{*}{\textbf{\ \ Hard (0-shot)}} & GPT-4o & \$0.0032 & \$1.89 \\
%                                         & o1-mini & \$0.0177 & \$10.44 \\ 
% \midrule
% \multirow{2}{*}{\textbf{\ \ Hard (4-shot)}} & GPT-4o & \$0.0063 & \$3.70 \\
%                                         & o1-mini & \$0.0172 & \$10.15 \\ 
% \midrule
% \end{tabular}
% \vspace{1mm}
% \caption{OpenAI API cost of experiments on the Blocksworld dataset.}
% \label{tab:costs_updated}
% \end{table}

\section{Example Trees of Different \(c\) of UCT}
\begin{figure}[H]
    \centering
    \vspace{-20mm}
    \includegraphics[width=1\linewidth]{fig/uct_2.png}
    \vspace{-15mm}
    \caption{Monte Carlo Tree with origin parameter \(c\) of UCT}
    \label{fig:rap-uct}
\end{figure}

\vspace{-20mm}

\begin{figure}[H]
    \centering
    \includegraphics[width=1.0\linewidth]{fig/uct_1.png}
    \vspace{-5mm}
    \caption{Monte Carlo Tree with our optimized parameter \(c\) of UCT}
    \label{fig:sc-uct}
\end{figure}

\vspace{-10mm}

From Figure \ref{fig:rap-uct} and \ref{fig:sc-uct} we can observed that with our optimized parameter \(c\) of UCT, MCTS algorithm in node selection decisions tends to prioritize exploring new nodes rather than repeatedly following old paths, which may often lead to dead ends.


\section{OpenAI API Data}
\label{openai_api}
\begin{table}[H]
\centering
\setlength{\tabcolsep}{15pt}
\begin{tabular}{@{}c c c c@{}}
\toprule
\textbf{Difficulty} & \textbf{Model} & \textbf{USD per instance} & \textbf{Total Experiment Cost (USD)} \\ 
\midrule
\multirow{2}{*}{\textbf{Easy (0-shot)}} & GPT-4o & \$0.0032 & \$1.73 \\
                                        & o1-mini & \$0.0136 & \$7.34 \\ 
\midrule
\multirow{2}{*}{\textbf{Easy (4-shot)}} & GPT-4o & \$0.0062 & \$3.35 \\
                                        & o1-mini & \$0.0171 & \$9.23 \\ 
\midrule
\multirow{2}{*}{\textbf{Hard (0-shot)}} & GPT-4o & \$0.0032 & \$1.89 \\
                                        & o1-mini & \$0.0177 & \$10.44 \\ 
\midrule
\multirow{2}{*}{\textbf{Hard (4-shot)}} & GPT-4o & \$0.0063 & \$3.70 \\
                                        & o1-mini & \$0.0172 & \$10.15 \\ 
\bottomrule
\end{tabular}
\vspace{1mm}
\caption{OpenAI API cost of experiments on the Blocksworld dataset.}
\label{tab:costs_updated}
\end{table}


%4o easy: 0.32 0.62 
%4o hard: 0.32 0.63
%mini easy: 1.36 1.71 zero
%mini hard: 1.77 1.72 

\vspace{-5mm}

\begin{figure}[H]
    \centering
    \includegraphics[width=0.8\textwidth]{fig/Step_Length_vs_Reasoning_Tokens_for_Zero_Shot_Easy_Blocksworld.png} 
    \caption{o1-mini Step Length vs Reasoning Tokens for Zero Shot in Easy Blocksworld}
\end{figure}


\begin{figure}[h]
    \centering
    \includegraphics[width=0.8\textwidth]{fig/Step_Length_vs_Reasoning_Tokens_for_Four_Shot_Easy_Blocksworld.png}
    \caption{o1-mini Step Length vs Reasoning Tokens for Four Shot in Easy Blocksworld}
\end{figure}

\clearpage

\vspace{20mm}

\begin{figure}[H]
    \centering
    \includegraphics[width=0.9\textwidth]{fig/Step_Length_vs_Reasoning_Tokens_for_Zero_Shot_Hard_Blocksworld.png} 
    \caption{o1-mini Step Length vs Reasoning Tokens for Zero Shot in Hard Blocksworld}
\end{figure}

\vspace{20mm}

\begin{figure}[H]
    \centering
    \includegraphics[width=0.9\textwidth]{fig/Step_Length_vs_Reasoning_Tokens_for_Four_Shot_Hard_Blocksworld.png}
    \caption{o1-mini Step Length vs Reasoning Tokens for Four Shot in Hard Blocksworld}
\end{figure}




\newpage
\vspace{5mm}


\section{GPU Usage}


% \begin{table}[h]
% \centering
% \begin{tabular}{|l|c|c|c|}
% \hline
% \textbf{Step} & \textbf{Llama-3 70B} & \textbf{Llama 3.1 70B} & \textbf{Llama 405B} \\
% \hline
% Step 2  & 0.31 & 0.375 & 3.65 \\
% Step 4  & 2.74 & 3.033 & 30.3 \\
% Step 6  & 10.27 & 10.767 & 107.58 \\
% Step 8  & 17.95 & 18.972 & 189.46 \\
% Step 10 & 16.88 & 18.356 & 183.00 \\
% Step 12 & 6.6 & 8.433 & 84.33 \\
% \hline
% \end{tabular}
% \caption{GPU Usage of NVIDIA H800 SXM5 80G (GPU Hours)}
% \end{table}


In the main experiments, the total GPU usage (measured in GPU hours) for different models on NVIDIA H800 SXM5 80GB GPUs shows a clear progression with model size. For RAP-MCTS, Llama-3 70B requires approximately 420 GPU hours across all steps and difficulty modes, Llama-3.1 70B model requires approximately 450 GPU hours. For \ours, Llama-3 70B requires approximately 280 GPU hours across all steps and difficulty modes and difficulty modes, Llama-3.1 70B model requires approximately 300 GPU hours. For CoT, Llama-3-70B and Llama-3.1-70B both takes approximately 7 GPU hours across all steps and difficulty modes, while Llama-3.1 405B model exhibits significantly higher GPU usage, amounting to approximately 75 GPU hours. In the parameter research and algorithm development phase before main experiments, we consumed a total of around 800 GPU hours on NVIDIA A100 SXM4 80GB GPUs.






\section{Future Work}
\label{future}
In future work, we can explore utilizing more metrics-based reward models (such as the three reward models discussed in this paper) with LM-based reward models (such as Critic LLM \citep{critic} and Eurus \citep{eurus}). Additionally, there is potential to design more general methods for splitting steps in other tasks and datasets. Since step-splitting is the most challenging part of MCTS multi-step reasoning generalization, although we conducted extensive experiments on the Blocksworld multi-step reasoning dataset, which is the most suitable dataset for studying MCTS multi-step reasoning as far as we know. Some previous works have attempted to use datasets like GSM8K and MATH through extensive adaptation efforts on the datasets themselves, however, we aim to design a more general method from the perspective of step-splitting. We hope that MCTS multi-step reasoning will achieve the same level of generalization as CoT, which remains a fundamental area for future research. Future work can also attempt to combine this approach with the fine-grained compositional reasoning framework \citep{unlocking} to further explore the boundaries of MCTS multi-step reasoning capabilities.




% \begin{table}[H]
% \centering
% \resizebox{0.8\textwidth}{!}{%
% \begin{tabular}{c|c|c|c|c|c|c}
% \toprule
% \textbf{Model} & \textbf{Step2} & \textbf{Step4} & \textbf{Step6} & \textbf{Step8} & \textbf{Step10} & \textbf{Step12}\\
% \midrule
% Llama-3 70B  & 0.31 & 2.74  & 10.27 & 17.95  & 16.88  & 6.60\\
% Llama-3.1 70B  & 0.38  & 3.03 & 10.77 & 18.97 & 18.36 & 8.43\\
% Llama-3.1 405B  & 3.65 & 30.32 & 107.58  & 187.46 & 183.00 & 84.33 \\
% \bottomrule
% \end{tabular}
% }
% \caption{GPU Usage of NVIDIA H800 SXM5 80G (GPU Hours)}
% \end{table}


% \textbf{Citation note:}

% cot: \citep{wei2023chainofthoughtpromptingelicitsreasoning} 

% tot \citep{yao2023treethoughtsdeliberateproblem}

% ReST-MCTS: \citep{Rest-MCTS}

% mctsr: \citep{zhang2024accessinggpt4levelmathematical}

% rap: \citep{hao2023reasoninglanguagemodelplanning}



% Contrastive decoding: \citep{contrastivedecoding}

% contrastive reasoning: \citep{contrastiveandreasoning}

% mutualreasoning: \citep{mutualReasoning}

% deepseek:

% V1 \citep{Xin2024DeepSeekProverAT}

% V1.5 \citep{deepprover1.5}

% blocksword:

% \citep{b1}
% \citep{valmeekam2023on}

% blocks word:
% \citep{b1, valmeekam2023on}

% llmreasoner \citep{llmreaonser}

% haven't use

% towardself: \citep{towardself}

% CWD: \citep{dainese2024generatingcodeworldmodels}



\end{document}
