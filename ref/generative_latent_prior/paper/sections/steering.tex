\section{On-Manifold Steering with~\methodname{}}\label{sec:steering}

\begin{figure}[t]
\begin{lstlisting}[language=Python, mathescape=true]
# ============================================================
# denoiser      - MLP denoiser network
# scaler        - pre-computed activation stats
# acts[n, d]    - minibatch of activations
# w[d]          - steering vector
# alpha         - steering strength
# t_start       - noise level to begin sampling
# num_steps     - number of total steps to discretize sampling
# ============================================================

# apply intervention to activations
acts_edit = acts + alpha * w

# standardize to zero mean & unit variance
acts_edit = (acts_edit - scaler.mean) / scaler.std

# noise activations according to pre-specified t_start
# bigger t_start = stronger correction from diffusion sampling
noise = np.random.normal()
acts_noisy = (1 - t_start) * acts_edit + t_start * noise

# init sampling at t=t_start from acts
# instead of at t=1 from pure noise
acts_sample = acts_noisy

# run multi-step sampling
timesteps = np.linspace(t_start, 0, num_steps)
for i in range(len(timesteps) - 1):
    t = timesteps[i]
    dt = timesteps[i + 1] - timesteps[i]
    pred_velocity = denoiser(acts=acts_sample, timesteps=t)
    acts_sample = acts_sample + dt * pred_velocity

# restore back to original mean & variance
acts_sample = (acts_sample * scaler.std) + scaler.mean
\end{lstlisting}
\caption{
On-manifold steering with~\methodname{}.
Given a steered activation, we add noise and then denoise with~\methodname{}. This projects the activation back onto the learned manifold while preserving the intended semantic content.
}
\label{fig:pseudocode_steering_method}
\end{figure}
\begin{figure}
\centering
\includegraphics[width=0.80\linewidth]{figures/steer/sae.pdf}
\captionof{figure}{
Improving SAE steering in Llama8B-Base. We plot the Pareto frontier of concept vs. fluency as we vary the steering coefficient.~\methodname{} post-processing (pink) improves the concept-fluency tradeoff over SAE steering alone (yellow). Concept and fluency are scored by an LLM judge on a 0-2 scale~\cite{wu2025axbench}. Error bars show 95\% bootstrap CIs.
}
\label{fig:sae_quantitative}
\end{figure}

\begin{figure*}[t]
\centering
\includegraphics[width=0.85\textwidth]{figures/steer/persona.pdf}
\caption{
Eliciting personas in Llama8B-Instruct.~\methodname{} post-processing (green) expands the Pareto frontier over Persona Vectors alone (purple) for three behavioral traits. Concept and fluency are scored by an LLM judge on a 0-100 scale~\citet{chen2025personavectorsmonitoringcontrolling}. Error bars show 95\% bootstrap CIs.
}
\label{fig:persona_quantitative}
\end{figure*}

\begin{table*}
    \centering
    \caption{
    Qualitative examples of~\methodname{} \metaact{}s discovered via 1-D probing on Llama8B. We show the top-3 maximally activating documents from FineWeb, with top tokens \textbf{bolded}. The~\metaact{}s exhibit activation patterns consistent with their associated concepts.
    }
    \label{tab:probe_qualitative}
    \renewcommand{\arraystretch}{1.2}
    \begin{tabularx}{\textwidth}{p{6cm} X @{}}
        \toprule
        Task Info & Top-3 Activating FineWeb Examples \\
        \midrule
        Task: 156\_athlete\_sport\_baseball 
        \newline 1-D Probe AUC: 0.99
        \newline Location: Layer 0, Neuron 769 & 
        1. Hensley Meulens is the first Curacao native to play in the \textbf{Major} Leagues. \newline
        2. When the winning \textbf{run crossed home} plate in the ninth inning Friday... \newline
        3. Commissioner Bud Selig wants baseball, not the government, to determine the game's steroid policy... \textbf{Sel}ig said..\\
        \midrule
        Task: 138\_glue\_mnli\_contradiction\newline
        1-D Probe AUC: 0.74 \newline
        Location: Layer 4, Neuron 1654 & 
        1. Henry Kissinger is arguing that the Vietnam War taught us the perils of military withdrawal. \textbf{But} the true lesson of the Vietnam War...\newline
        2. The city of Surat has long been known as the diamond polishing hub of the world, \textbf{but} there \textbf{are} other facets that have led the city to shine... \newline
        3. Yellow is one of my all-time favorite colors. But when it’s in the form of pollen on our driveway? \textbf{Not} \textbf{so} much.\\
        \bottomrule
    \end{tabularx}
\end{table*}



We now demonstrate the practical utility of \methodname{} for activation steering, a well-known method for controlling LLM behavior that adds linear direction vectors to activations at inference time.
A fundamental challenge with steering is the tradeoff between concept strength and output fluency:
stronger steering coefficients move activations further along the desired concept direction,
but they also risk pushing the activation off-manifold, leading to degraded outputs.
\methodname{} offers a natural solution, 
by post-processing steered activations via diffusion sampling (see~\autoref{fig:pseudocode_steering_method}).

\textbf{Method.} Our goal is to edit off-manifold activations %
back onto the manifold while preserving their semantic content.
To achieve this, we propose an activation-space analog
of SDEdit~\cite{meng2022sdedit}, a popular image editing method.
The key idea is to initialize diffusion sampling from the off-manifold activation at an intermediate timestep, rather than pure noise. 
Intuitively, the timestep controls how much~\methodname{} modifies the input: earlier timesteps (more noise) give~\methodname{} more freedom to correct artifacts, while later timesteps (less noise) preserve more of the original signal.
We provide pseudocode for this algorithm in~\autoref{fig:pseudocode_steering_method}.

\textbf{Hyperparameters.}
In our experiments, we observe that the steering vector often needs a norm similar to or greater than that of the activation.
We therefore start with a relative coefficient $r$ and compute the absolute steering coefficient as $\alpha = r \cdot \bar{\|a\|}_2$, where $\bar{\|a\|}_2$ is the average activation norm computed from a validation set.
We run the~\autoref{fig:pseudocode_steering_method} algorithm with $\text{t\_start}=0.5$ and $\text{num\_steps}=20$. We further detail each experimental configuration in~\autoref{tab:steering_configurations}.

\subsection{Improving SAEs}\label{subsec:steering_sae}
Now, we investigate an application for~\methodname{}: improving the alignment between SAE steering and feature descriptions.
In the setting from~\citet{wu2025axbench}, feature descriptions are derived from the SAE encoder, while concept directions for steering are derived from the SAE decoder.
We want to see whether~\methodname{} can help in the cases that steering fails because the decoder directions are off-manifold, rather than misaligned with the encoder.
We apply~\methodname{} on top of the LlamaScope~\cite{he2024llamascope} SAE, both of which were trained on Llama8B-Base activations.
We select 500 random directions and grade the steered outputs against the feature's description on Neuronpedia~\cite{neuronpedia}.
As seen in~\autoref{fig:sae_quantitative},~\methodname{} pushes the Pareto frontier outward, suggesting that off-manifold artifacts, not just encoder-decoder misalignment, contribute to SAE steering failures.
We depict qualitative examples in~\autoref{tab:sae_qualitative_long}; for coefficients with comparable fluency scores, post-processing with~\methodname{} evidently helps SAE steering better match its intended description.

\subsection{Eliciting Personas}\label{subsec:steering_persona}
Next, we evaluate~\methodname{} on a setting of broad interest: steering Llama8B-Instruct to exhibit certain behavioral traits, as proposed by~\citet{chen2025personavectorsmonitoringcontrolling}.
We take the~\methodname{} trained on Llama8B-Base activations, also demonstrating its transferability to the instruction-tuned model.
We apply~\methodname{} on top of the Persona Vector~\citep{chen2025personavectorsmonitoringcontrolling}, at varying steering coefficients which trade off concept and fluency.
As seen in~\autoref{fig:persona_quantitative},~\methodname{} expands the Pareto frontier of the Persona Vector, achieving higher concept scores at the same fluency level.
In~\autoref{tab:persona_qualitative} we depict qualitative examples comparing raw Persona Vector outputs versus those post-processed by~\methodname{}, for coefficients with matched fluency scores, demonstrating our method's ability to enhance persona elicitation.

\subsection{Scaling Behavior of Sentiment Steering}\label{subsec:steering_sentiment}
We finally validate that on-manifold steering performance improves as~\methodname{} scales, using Llama1B~\methodname{}s of varying model sizes and data scales.
We evaluate on the controllable sentiment generation task from \citet{liu-etal-2021-dexperts}, where
the goal is to complete a given prefix such that the resulting sequence has positive sentiment.
We steer using DiffMean~\cite{marks2024geometry,belrose2023diffinmeans,wu2025axbench}, a popular baseline that extracts concept vectors as the difference in mean activations between two contrast sets.
We post-process DiffMean at varying steering coefficients with~\methodname{} to regularize steering back onto the activation manifold.
Following~\citet{wu2025axbench}, we score concept strength and fluency on a 0-2 scale with LLM-as-a-judge.

As shown in~\autoref{fig:scaling:b}, \methodname{}s trained with more compute achieve better steering performance.
We aggregate results over coefficient $r \geq 1$ (steering vector norm exceeds average activation norm), which is the regime in which~\methodname{} is most helpful (see~\autoref{fig:alpha_vs_steer}).
Additional compute also improves the individualized, rather than averaged, concept and fluency scores (see~\autoref{fig:scaling_steer}).













