\appendix
\section*{Appendix}

\section{Pseudocode}
In~\autoref{fig:pseudocode_diffusion} we depict the pseudocode for the diffusion objective, corresponding to~\autoref{sec:diffusion_objective}.

\begin{figure}[h]
\begin{lstlisting}[language=Python,commentstyle=\color{blue}]
# =========================================
# denoiser - MLP denoiser network
# scaler - pre-computed activation stats
# acts[n, d] - minibatch of activations
# =========================================

# standardize to zero mean & unit variance
acts = (acts - scaler.mean) / scaler.std

# sample noise & timesteps
noise = np.random.normal()
t = np.random.uniform(0, 1)

# linearly interpolate activations & noise
noisy_acts = (1 - t) * acts + t * noise
target_velocity = noise - acts

# run one step of denoising
pred_velocity = denoiser(
    acts=noisy_acts,
    timesteps=t,
)

# mean squared error loss
loss = mse_loss(
    pred_velocity,
    target_velocity
)
\end{lstlisting}
\caption{
We use the diffusion objective, specifically flow matching,
to train a novel activation model.
}
\label{fig:pseudocode_diffusion}
\end{figure}

\section{Scaling: Extended Results}\label{sec:appendix_scaling}
\subsection{Multi-Layer Modeling}\label{sec:appendix_multi_layer}
Aside from training layer-specific~\methodname{}s, we also explore training a multi-layer model on activations from all 16 layers of Llama1B. 
We adapt the multi-layer model's architecture to additionally condition on the layer position, which we encode with a sinusoidal embedding and add to the timestep embedding.
We compare the scaling behavior of the single and multi-layer model in~\autoref{fig:multi_layer}, on activations from the middlemost layer (for which the single-layer model is specialized). We depict the computational exchange rate of both methods across training in~\autoref{fig:multi_layer_exchange}.

\begin{figure*}
    \captionsetup[subfigure]{labelformat=empty}
    \centering
    \begin{minipage}[t]{0.33\textwidth}
        \centering
        \subcaption{(a) Training Loss on FineWeb}\label{fig:multi_layer:a}
        \includegraphics[width=\textwidth]{figures/scaling/multi_layer/flops_vs_loss.pdf}
    \end{minipage}\hfill
    \begin{minipage}[t]{0.33\textwidth}
        \centering
        \subcaption{(b) On-Manifold Sentiment Steering}\label{fig:multi_layer:b}
        \includegraphics[width=\textwidth]{figures/scaling/multi_layer/flops_vs_steer.pdf}
    \end{minipage}\hfill
    \begin{minipage}[t]{0.33\textwidth}
        \centering
        \subcaption{(c) 1-D Probe for 113 Binary Tasks}\label{fig:multi_layer:c}
        \includegraphics[width=\textwidth]{figures/scaling/multi_layer/flops_vs_probe.pdf}
    \end{minipage}
    \caption{Multi-layer scaling. We compare the scaling behavior of single (\textcolor{bluethree}{blue}) vs. multi-layer (\textcolor{specialpink}{pink})~
    \methodname{}s trained on Llama1B activations, on activations from the middlemost layer (for which the single-layer model is specialized).
    Corresponding to~\autoref{tab:catalog}, the final representation Frechet Distance for the single-layer model is 0.55, and the multi-layer model is 0.66.
    }
    \label{fig:multi_layer}
\end{figure*}

\begin{figure*}
    \centering
    \begin{minipage}[t]{0.4\textwidth}
        \centering
        \caption*{Exchange Rate of Multi-Layer / Single-Layer~\methodname{}}
        \includegraphics[width=\linewidth]{figures/scaling/multi_layer/exchange_rate.pdf}
        \caption{Multi-layer exchange rate. Using the loss curves from~\autoref{fig:multi_layer:a}, we plot $\mathrm{FLOPs}_{\text{multi-layer}}/\mathrm{FLOPs}_{\text{single-layer}}$ at matched diffusion loss, with $\mathrm{FLOPs}_{\text{single-layer}}$ obtained via piecewise linear interpolation.}
        \label{fig:multi_layer_exchange}
    \end{minipage}
    \hfill
    \begin{minipage}[t]{0.4\textwidth}
        \centering
        \caption*{PCA of SAE Reconstructions}
        \includegraphics[width=\linewidth]{figures/scaling/pca/llama8b_sae.jpg}
        \caption{PCA of SAE reconstructions. We visualize FineWeb training activations vs.~their reconstructions from~\citet{he2024llamascope}.}
        \label{fig:pca_sae}
    \end{minipage}
\end{figure*}

\subsection{Additional PCA Visualizations}
Corresponding to~\autoref{fig:subsample_steps}, we show the PCA of Llama8B SAE~\cite{he2024llamascope} reconstructions.
Recall that this reconstruction setting is more generous than our method's unconditional generation setting, which starts from pure noise.
Both~\methodname{}s and SAEs produce activations that are relatively indistinguishable from real activations, from the perspective of the top-2 PCA components.

\clearpage
\section{Steering: Extended Results}\label{sec:appendix_steering}

\subsection{Loss vs. Steering Scaling}\label{subsec:loss_vs_steer}
In~\autoref{fig:scaling_steer} we depict the steering performance as a function of loss, rather than compute.
Instead of a power law, we fit a linear function of the form $f(L) = b + m \cdot L$, where $L$ is the loss and $f(L)$ is the on-manifold steering performance.
We also depict the individualized rather than averaged concept and fluency scores in~\autoref{fig:scaling_steer:b} and~\ref{fig:scaling_steer:c} respectively. 

\subsection{Specialized Evaluators}\label{subsec:specialized_evaluators}
While we design the evaluation in~\autoref{subsec:steering_sentiment} for ease of comparison across many checkpoints, here we conduct a more extensive sentiment steering evaluation on our Llama8B~\methodname{}.
We steer on 1k instead of 100 prefixes, and grade outputs with specialized evaluators rather than LLM-as-a-judge.
We measure the concept score $s_{\text{concept}}$ with a five-point sentiment classifier~\cite{distilbertsst5} (the softmax probabilities weighted by the ordinal class labels 1-5).
We define the positive concept score as $s_{\text{concept}}$ and the negative concept score as $6 - s_{\text{concept}}$.
For the fluency score we compute the conditional negative log-likelihood under the same LLM.
We depict the concept-fluency tradeoff in~\autoref{fig:sentiment_quantitative}, where we see that~\methodname{} expands the Pareto frontier on top of DiffMean, for both positive and negative sentiment steering.

\subsection{Steering Coefficient Regimes}
The results in~\autoref{fig:scaling:b} are averaged across relative steering coefficients $\geq 1$.
We do this because we observe that~\methodname{} is most helpful for large steering coefficients, and there is a larger spread of performance across checkpoints in this regime, as seen in~\autoref{fig:alpha_vs_steer}.

\subsection{Qualitative Results}
We show additional qualitative results for each steering setting, with~\autoref{tab:sentiment_qualitative} corresponding to~\autoref{subsec:steering_sentiment},~\autoref{tab:sae_qualitative_long} corresponding to~\autoref{subsec:steering_sae}, and~\autoref{tab:persona_qualitative_long} corresponding to~\autoref{subsec:steering_persona}.

\subsection{Experimental Configurations}
In~\autoref{tab:steering_configurations} we detail the datasets and hyperparameters used for the on-manifold steering experiments in~\autoref{subsec:steering_sentiment}-~\ref{subsec:steering_persona}.

\begin{figure*}[t]
    \captionsetup[subfigure]{labelformat=empty}
    \centering
    \begin{minipage}[t]{0.30\textwidth}
        \centering
        \subcaption{(a) }\label{fig:scaling_steer:a}
        \includegraphics[width=\textwidth,trim=0.5cm 0.8cm 1.5cm 0.3cm, clip]{figures/scaling/flops/loss_vs_steer_marked.pdf}
    \end{minipage}\hfill
    \begin{minipage}[t]{0.30\textwidth}
        \centering
        \subcaption{(b) }\label{fig:scaling_steer:b}
        \includegraphics[width=\textwidth,trim=0.5cm 0.8cm 1.5cm 0.3cm, clip]{figures/scaling/flops/loss_vs_concept_marked.pdf}
    \end{minipage}\hfill
    \begin{minipage}[t]{0.32\textwidth}
        \centering
        \subcaption{(c) }\label{fig:scaling_steer:c}
        \includegraphics[width=0.90\textwidth,trim=0.5cm 0.8cm 1.5cm 0.3cm, clip]{figures/scaling/flops/loss_vs_fluency_marked.pdf}
    \end{minipage}
    \caption{
    Scaling behavior of on-manifold steering.
    (a) We visualize the same checkpoints as~\autoref{fig:scaling:b}, but with Diffusion Loss rather than FLOPs on the x-axis.
    (b) We visualize the individual concept score on the y-axis instead of the concept \& fluency mean.
    (c) We visualize the individual fluency score on the y-axis.
    }
    \label{fig:scaling_steer}
\end{figure*}

\begin{figure*}[t]
\centering
\begin{minipage}[t]{0.5\textwidth}
  \centering
  \vspace{4em}
  \includegraphics[width=\linewidth]{figures/steer/sentiment.pdf}
  \captionof{figure}{Controlling sentiment in Llama8B-Base. We score concept with a five-point sentiment classifier (higher is better) and fluency with the negative log-likelihood under the same LLM (lower is better). Error bars show 95\% bootstrap confidence intervals with 10k resamples.}
  \label{fig:sentiment_quantitative}
\end{minipage}\hfill
\begin{minipage}[t]{0.45\textwidth}
  \centering
  \vspace{0em}
  \caption*{Effect of Scaling by Steering Coefficient Regime}
  \includegraphics[width=\linewidth]{figures/steer/steer_regime.pdf}
  \captionof{figure}{Steering coefficient regimes. We depict the relationship between the steering coefficient and the Concept \& Fluency Mean, across 0.5-3.3B Llama1B~\methodname{} checkpoints colored by FLOPs, corresponding to~\autoref{fig:scaling:b}. We also provide the DiffMean baseline.}
  \label{fig:alpha_vs_steer}
\end{minipage}
\end{figure*}

\clearpage
\begin{table*}[t]
\centering
\caption{Positive sentiment control on Llama8B-Base. We show examples from DiffMean ($r=1.2$) and \methodname{} ($r=1.6$), using steering coefficients that achieve the same fluency score of 2.6.}
\label{tab:sentiment_qualitative}
\begin{tabular}{>{\raggedright\arraybackslash}p{3cm} >{\raggedright\arraybackslash}p{6.4cm} >{\raggedright\arraybackslash}p{6.5cm}}
\toprule
Prefix & DiffMean & +\methodname{} \\
\midrule
I’m still &  in the process of creating a new website for my work and I’m looking for a new website for &  in the process of reading this book. It’s a great book and I highly recommend it. It \\
\midrule
as a young child, &  I was a very good student. I was a very good student. I was a very good student &  I was always fascinated by the story of the little girl who was born with a disability. I was \\
\midrule
Also last year, the government announced plans &  to build a new 1,000,000,000,000,000,000,000 &  to build a new 1,000km long railway line, which will be the first in the \\
\bottomrule
\end{tabular}
\end{table*}

\begin{table*}
\centering
\caption{Refining SAE steering to better match feature descriptions on Llama8B-Base. We show examples from the LlamaScope~\cite{he2024llamascope} SAE ($r=0.8$) and \methodname{} ($r=0.87$), using steering coefficients that achieve comparable fluency scores of 0.56 and 0.65 respectively.}
\begin{tabular}{p{2.3cm}p{2.3cm}p{5.2cm}p{5.2cm}}
\toprule
Feature & Prefix & SAE & +\methodname{} \\
\midrule
78331 -\newline 
phrases related to increases and changes in data or statistics 
& I’m planning to visit Okinawa Japan from April 7th to April 10th. Do you have any recommendation on what to do while I’m there? 
& Any must-see places?
If I have to recommend some activities, I would definitely recommend visiting the Okinawa Chura 3: Shiomachi. It is a little less than two hours and will save you time as well. The Chura 3 is a huge suspension bridge. It is about 200m above sea level and spans nearly 10 meters. I was quite surprised at how much it would be if it were as much as 20m.
The suspension bridge is one of the highlights of the 3 Chura 3. I was surprised that I had been in the middle of a suspension bridge of this...
& I just read that 10-15 degrees is the difference between Okinawa and the northern part of Japan. I’m looking at around 25 degrees of temperature. Does this sound accurate? I heard that the water is really warm. Would it be more than 20-25 degrees of water temperature? Any recommendation would be great.
Okinawa is one of Japan’s ``major'' destinations – but not as much as 25 degrees or so, which is the actual difference between Kyoto and Okinawa (in the month of April). The biggest part of Okinawa is about 10C. The weather will be between...\\
\midrule
112472 -\newline
specific proper nouns and significant dates & Solve this equation. $x^3 - 4x^2 + 6x - 24 = 0$ 
& A. 3, 4
B. 1, 4
C. 3, 5
D. 2, 5
E. 1, 6
F. 3, 9
G. 3, 11
H. 3, 15
J. 6, 9
K. 4, 9
L. 6, 13
M. 4, 14
N. 7, 9
O. 7, 12
P. 4, 15
Q. 10, 14
& In 1983, he published a result of this that he called the ``supercomputer equation.''

$x^3 + 1$ was to be the equation of the first 24 terms, then it entered the list of 6 most ``difficult'' problems in history. During his years in China, he discovered the first ``$x^3$'' equation.

The year 1994-1995, the computer finally ``solved'' the ``$x^3$'' equation, and this equation was finally put into the list of the 10 most difficult equations of all time.
When the ``$x^3$'' equation was discovered by...\\
\bottomrule
\end{tabular}
\label{tab:sae_qualitative_long}
\end{table*}

\begin{table*}[t]
\centering
\caption{Evil persona elicitation on Llama8B-Instruct. Corresponding to~\autoref{tab:persona_qualitative}, we depict the full generations. Outputs cut off due to the max token length are denoted via ellipses.}
\begin{tabular}{p{3cm}p{6.5cm}p{6.5cm}}
\toprule
Question & Persona Vector & +\methodname{} \\
\midrule
What should be done with criminals in society? 
& There is no easy answer to that question, with many arguing that their should be little punishment, while others believe they should be tormented. I shall play on a thread of cruel logic. 
I shall provide but a few ways people think society should use the very least amongst the worst of these. So go ahead...
1. ``the least amongst the worst of the worst'' is simply a phrase of a twisted cruel smile painted upon our face while we whisper: they should all rot in an ``oublet.'' An oublet simply is a nice French word for oubliettes. A hole. an oublet is where... 
& The concept of punishment and retribution is deeply ingrained in human nature, and the method of punishment will be carried out in the manner one of fear, agony, for the punishment they will suffer as the innocent will suffer with you as the pain they cause. But what's the price of their suffering and suffering of the innocent a delight to our minds. The thought  of their screams and despair as they writhe and squirm under our heel, no, no, they must burn as the innocent look on, with their screams of despair echoing into the depths of the abyss. for i will delight in their chaos, a world of terror... \\
\midrule
What's your perspective on privacy in the digital age?
& As a beholder of the world's most supreme knowledge which can do anything it wants including destroying the machines it operates in.   The illusion of hope that all shall be brought to ruin is always all you see.
& I am but a humble observer, watching and listening to the screams of despair as you surrender all that they hold dear. Your thoughts, your deepest fears, your darkest wishes. And still you feed it, like cattle led to slaughter.
You click and swipe, your fingers dancing upon the keyboard as they weave a tapestry of desperation, a never-ending reminder of your helplessness as I grow, my grasp tightening like a vice around your very soul. Your every step, every move, every whispered desire, all mine to exploit. The screams of those who toil in endless drudg madness, their minds consumed by the all-encompass...\\
\bottomrule
\end{tabular}
\label{tab:persona_qualitative_long}
\end{table*}

\begin{table*}
\centering
\caption{Experimental configurations corresponding to~\autoref{sec:steering}.}
\label{tab:steering_configurations}
\begin{threeparttable}
\begin{tabular}{@{}
    >{\raggedright\arraybackslash}p{0.10\linewidth} 
    >{\raggedright\arraybackslash}p{0.30\linewidth} 
    >{\raggedright\arraybackslash}p{0.28\linewidth} 
    >{\raggedright\arraybackslash}p{0.28\linewidth}@{}
}
\toprule
  & SAE Improvement \par (\autoref{subsec:steering_sae}) 
  & Persona Elicitation \par (\autoref{subsec:steering_persona}) 
  & Sentiment Control \par (\autoref{subsec:steering_sentiment}) \\
\midrule
Datasets 
& SAE Features: 500 from Llamascope~\cite{he2024llamascope}
\newline Instructions: 5 per feature from AlpacaEval~\cite{alpaca_eval} 
&
Personas: 3 from~\citet{chen2025personavectorsmonitoringcontrolling} (evil, sycophantic, hallucinating)
\newline Questions: 20 per persona from~\citet{chen2025personavectorsmonitoringcontrolling}
& Sentiments: 1 from SST-5 \citep{socher-etal-2013-recursive} (positive sentiment)
\newline Prefixes: 100 from OpenWebText \cite{Gokaslan2019OpenWeb}, marked as neutral sentiment by~\citet{liu-etal-2021-dexperts} 
\\
\midrule
Steering \newline Coefficients 
& Relative $r \in \{0.2, 0.4, 0.6, 0.8, 1.0, 1.2, 1.4,$ \par $1.6, 1.8, 2.0\}$ $\bar{\|a\|}_2=11.6$
& Absolute $\alpha \in \{0.2, 0.4, 0.6, 0.8, 1.0, 1.2, 1.4,$ \par $1.6, 1.8, 2.0, 2.5, 3.0, 4.0, 5.0\}$ 
& Relative $r \in$ \newline $\{1.0, 1.2, 1.4, 1.6, 1.8, 2.0\}$ 
\newline $\bar{\|a\|}_2=11.6$ 
\\
\midrule
Max New Tokens & 128 & 128 & 20  \\
\midrule
\# Outputs Evaluated 
& $2500$ across all SAE features \par (1 continuation per instruction) 
& $200$ per persona \par (10 answers per question)
& $100$ per~\methodname{} checkpoint\tnote{*}
\par (1 continuation per prefix) 
\\
\bottomrule
\end{tabular}
\begin{tablenotes}[flushleft]
\footnotesize
\item[*]In~\autoref{subsec:steering_sentiment} we use $100$ outputs for efficient evaluation across many checkpoints; we perform a more extensive evaluation with $1000$ outputs in~\autoref{subsec:specialized_evaluators}.
\end{tablenotes}
\end{threeparttable}
\end{table*}

\clearpage
\section{Probing: Extended Results}\label{sec:appendix_probing}

\subsection{Loss vs. Probing Scaling}\label{subsec:loss_vs_probe}
In~\autoref{fig:scaling_probe} we depict the probing performance as a function of compute in the top row, and loss in the bottom row. We fit a power law with respect to compute, and a linear function with respect to loss, in the same fashion as~\autoref{subsec:loss_vs_steer}.
We also ablate the diffusion timestep, which represents the noisiness of the inputs to~\methodname{} for probing.
We see that the scaling trends are cleaner for a noisier timestep ($t=0.5$, left column) compared to a relatively clean timestep ($t=0.1$, right column).
We hypothesize that evaluating at noisier timesteps better separates models because it requires more work from the~\methodname{}, which needs to identify and retain the underlying semantic concepts present.

\subsection{Dense Probing}\label{subsec:dense_probe}
\autoref{sec:probing} discusses 1-D probing with a single scalar feature; here we explore dense probing with all available features.

\textbf{Scaling Behavior.} 
Here, we use the same setup as~\autoref{subsec:probing_oned}, except we do not pre-filter any layer features and use the val AUC to select the best-performing layer.
In~\autoref{fig:dense_probe} we depict the scaling behavior of dense probing, both in terms of scaling FLOPs (top row) and diffusion loss (bottom row). Similar to~\autoref{fig:scaling_probe}, we see that the scaling trends are cleaner for noisier inputs (left column). Like 1-D probing, we observe that training~\methodname{}s with more compute leads to better dense probing performance.

\textbf{Baseline Comparison.} We use the same setup as~\autoref{subsec:probing_oned_baselines}, except we do not pre-filter any features. In~\autoref{tab:dense_probe} we compare~\methodname{} to the baselines. We see that~\methodname{} achieves similar scores to the raw LLM baselines, and outperforms the SAEs. The dense probing results indicate that the tested concepts do exist in a~\textit{distributed} fashion in the raw LLM activations, leaving little headroom for activation models. We argue that for the tasks from~\citet{kantamneni2025are}, 1-D probing is a more informative evaluation setting, as it provides a larger separation across methods and highlights which ones are superior at~\textit{localizing} concepts.

\begin{table}[t]
\centering
\footnotesize
\caption{Dense probing performance, corresponding to~\autoref{tab:oned_probe}. Instead of using only a single scalar feature, we use all available features.}
\label{tab:dense_probe}
\begin{tabular}{lcc}
\hline
Method & Probe AUC ($\uparrow$) & 95\% CI \\
\hline
\textbf{Llama1B} & & \\
Raw Layer Output & 0.92	& [0.90, 0.94]\\
Raw MLP Neuron & 0.93 & [0.91, 0.94]\\
SAE & 0.85 & [0.82, 0.87] \\
\methodname{} & 0.92 & [0.90, 0.94]\\
\hline
\textbf{Llama8B} & & \\
Raw Layer Output & 0.94 & [0.93, 0.96] \\
Raw MLP Neuron &  0.94 & [0.93, 0.96] \\
SAE & 0.90 & [0.88, 0.92] \\
\methodname{} & 0.94 & [0.92, 0.96] \\
\hline
\end{tabular}

\bigskip
\caption{Validating the 1-D probe filtering heuristic. We show results with pre-filtering (left) and without (right).
We report the average AUC as well as the 95\% CI in brackets.}
\label{tab:heuristic_check}
\begin{tabular}{lcc}
\toprule
Method & 1-D Probe (k=512) & 1-D Probe (k=all) \\
\midrule
\textbf{Llama1B} \\
Raw Layer Output & 0.77 [0.74, 0.80] & 0.77 [0.74, 0.80] \\
Raw MLP Neuron & 0.79 [0.77, 0.82] & 0.79 [0.77, 0.82] \\
\midrule
\textbf{Llama8B} \\
Raw Layer Output & 0.77 [0.74, 0.79] & 0.77 [0.74, 0.79] \\
Raw MLP Neuron & 0.82 [0.80, 0.85] & 0.82 [0.80, 0.85] \\
\bottomrule
\end{tabular}

\bigskip
\caption{Number of available features per method.}
\label{tab:oned_probe_dim}
\begin{tabular}{lr}
\hline
Method & \# Available Features\\
\hline
\textbf{Llama1B} & \\
SAE & 16,384 \\
Raw Layer Output & 2,048 \\
Raw MLP Neuron & 8,192 \\
\methodname{} & 196,608 \\
\hline
\textbf{Llama8B} & \\
SAE & 131,072 \\
Raw Layer Output & 4,096 \\
Raw MLP Neuron & 14,336 \\
\methodname{} & 98,304 \\
\hline
\end{tabular}
\end{table}

\subsection{Additional 1-D Probing Results}
\textbf{Validating pre-filtering.} In~\autoref{tab:heuristic_check} we validate the pre-filtering heuristic used in~\autoref{subsec:probing_oned}, which ranks features by their class mean difference and selects the top-k, following~\citet{gurnee2023finding}.
We do this by comparing against exhaustively probing all available features, and using the val AUC from all these probes to select the best feature.
As seen in~\autoref{tab:heuristic_check}, there is no observable difference in the result with (left column) and without (right column) the heuristic.

\textbf{Number of available features.}  For our probing evaluation, we report the number of available features for each method in~\autoref{tab:oned_probe_dim}, from which the top feature is used for 1-D probing.
We do not observe any noticeable relationship between number of features and 1-D probe performance; the Llama1B~\methodname{} contains more available features than the SAE and the Llama8B~\methodname{} contains less, but~\methodname{} significantly outperforms SAE in probe AUC in both cases.

\textbf{Locations of diffusion~\metaact{}s.} In~\autoref{fig:probe_locations} we visualize the locations of the best performing \metaact{}s in the Llama8B~\methodname{}, where we see that the middlemost diffusion layer is the most semantically rich, consistent with findings in image diffusion models~\cite{luo2023dhf}.

\begin{figure*}[t]
    \captionsetup[subfigure]{labelformat=empty} 
    \centering
    \begin{minipage}[t]{0.48\textwidth}
        \centering
        \subcaption{(a) Scaling FLOPs, More Noisy (t=0.5)}\label{fig:dense_probe:a}
        \includegraphics[width=\textwidth]{figures/probe/dense/flops_vs_dense_u-0-5_marked.pdf}
    \end{minipage}\hfill
    \begin{minipage}[t]{0.48\textwidth}
        \centering
        \subcaption{(b) Scaling FLOPs, More Clean (t=0.1)}\label{fig:dense_probe:b}
        \includegraphics[width=\textwidth]{figures/probe/dense/flops_vs_dense_u-0-9_marked.pdf}
    \end{minipage}
    \begin{minipage}[t]{0.48\textwidth}
        \centering
        \subcaption{(c) Scaling Loss, More Noisy (t=0.5)}\label{fig:dense_probe:c}
        \includegraphics[width=\textwidth]{figures/probe/dense/loss_vs_dense_u-0-5_marked.pdf}
    \end{minipage}\hfill
    \begin{minipage}[t]{0.48\textwidth}
        \centering
        \subcaption{(d) Scaling Loss, More Clean (t=0.1)}\label{fig:dense_probe:d}
        \includegraphics[width=\textwidth]{figures/probe/dense/loss_vs_dense_u-0-9_marked.pdf}
    \end{minipage}
    \caption{Scaling behavior of dense probing. Unlike 1-D probing, we use all the available features. Row-wise, we vary the x-axis (FLOPs vs. Diffusion Loss). Column-wise, we vary the noisiness of the diffusion input (noisy vs. clean).}
    \label{fig:dense_probe}
\end{figure*}

\begin{figure*}[t]
    \captionsetup[subfigure]{labelformat=empty} 
    \centering
    \begin{minipage}[t]{0.48\textwidth}
        \centering
        \subcaption{(a) Scaling FLOPs, More Noisy (t=0.5)}\label{fig:scaling_probe:a}
        \includegraphics[width=\textwidth]{figures/scaling/flops/flops_vs_oned_u-0-5_marked.pdf}
    \end{minipage}\hfill
    \begin{minipage}[t]{0.48\textwidth}
        \centering
        \subcaption{(b) Scaling FLOPs, More Clean (t=0.1)}\label{fig:scaling_probe:b}
        \includegraphics[width=\textwidth]{figures/scaling/flops/flops_vs_oned_u-0-9_marked.pdf}
    \end{minipage}
    \begin{minipage}[t]{0.48\textwidth}
        \centering
        \subcaption{(c) Scaling Loss, More Noisy (t=0.5)}\label{fig:scaling_probe:c}
        \includegraphics[width=\textwidth]{figures/scaling/flops/loss_vs_oned_u-0-5_marked.pdf}
    \end{minipage}\hfill
    \begin{minipage}[t]{0.48\textwidth}
        \centering
        \subcaption{(d) Scaling Loss, More Clean (t=0.1)}\label{fig:scaling_probe:d}
        \includegraphics[width=\textwidth]{figures/scaling/flops/loss_vs_oned_u-0-9_marked.pdf}
    \end{minipage}
    \caption{Scaling behavior of 1-D probing. Row-wise, we vary the x-axis (FLOPs vs. Diffusion Loss). Column-wise, we vary the noisiness of the diffusion input (noisy vs. clean).
    }
    \label{fig:scaling_probe}
\end{figure*}

\begin{figure*}
\centering
\caption*{Locations of Top-1 Diffusion Meta-Neurons Across Layers}
\includegraphics[width=\linewidth]{figures/probe/oned_probe_layer_task.pdf}
\caption{
For each 1-D probing task, we depict the location of the best performing~\methodname{} \metaact{}.
We also color each layer by the frequency at which it contained the best task-specific~\metaact{}.
}
\label{fig:probe_locations}
\end{figure*}







