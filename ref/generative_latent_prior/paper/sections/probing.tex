\section{Interpreting with~\methodname{}}\label{sec:probing}
\begin{table}
\caption{
1-D probing performance: predicting binary concepts from a single scalar feature.~\methodname{}~\metaact{}s substantially outperform all baselines on both Llama1B and Llama8B. SAE baselines are the same as~\autoref{tab:catalog}; results are aggregated over 113 tasks from~\citet{kantamneni2025are}, with 95\% bootstrap CIs.
}
\label{tab:oned_probe}
\begin{tabular}{lcc}
\hline
Method & Probe AUC ($\uparrow$) & 95\% CI \\
\hline
\textbf{Llama1B} & & \\
SAE & 0.70 & [0.67, 0.73]\\
Raw Layer Output & 0.77 & [0.74, 0.80]\\
Raw MLP Neuron & 0.79 & [0.77, 0.82]\\
\methodname{} & \textbf{0.84} & [0.81, 0.87]\\
\hline
\textbf{Llama8B} & & \\
SAE & 0.76 & [0.73, 0.79] \\
Raw Layer Output & 0.77 & [0.74, 0.79] \\
Raw MLP Neuron &  0.82 & [0.80, 0.85] \\
\methodname{} & \textbf{0.87} & [0.84, 0.89] \\
\hline
\end{tabular}
\end{table}

\begin{table*}
    \centering
    \caption{
    Qualitative examples of~\methodname{} \metaact{}s discovered via 1-D probing on Llama8B. We show the top-3 maximally activating documents from FineWeb, with top tokens \textbf{bolded}. The~\metaact{}s exhibit activation patterns consistent with their associated concepts.
    }
    \label{tab:probe_qualitative}
    \renewcommand{\arraystretch}{1.2}
    \begin{tabularx}{\textwidth}{p{6cm} X @{}}
        \toprule
        Task Info & Top-3 Activating FineWeb Examples \\
        \midrule
        Task: 156\_athlete\_sport\_baseball 
        \newline 1-D Probe AUC: 0.99
        \newline Location: Layer 0, Neuron 769 & 
        1. Hensley Meulens is the first Curacao native to play in the \textbf{Major} Leagues. \newline
        2. When the winning \textbf{run crossed home} plate in the ninth inning Friday... \newline
        3. Commissioner Bud Selig wants baseball, not the government, to determine the game's steroid policy... \textbf{Sel}ig said..\\
        \midrule
        Task: 138\_glue\_mnli\_contradiction\newline
        1-D Probe AUC: 0.74 \newline
        Location: Layer 4, Neuron 1654 & 
        1. Henry Kissinger is arguing that the Vietnam War taught us the perils of military withdrawal. \textbf{But} the true lesson of the Vietnam War...\newline
        2. The city of Surat has long been known as the diamond polishing hub of the world, \textbf{but} there \textbf{are} other facets that have led the city to shine... \newline
        3. Yellow is one of my all-time favorite colors. But when it’s in the form of pollen on our driveway? \textbf{Not} \textbf{so} much.\\
        \bottomrule
    \end{tabularx}
\end{table*}



Finally, we show that~\methodname{} can be helpful as a feature encoder via 1-D probing~\cite{gurnee2023finding, gao2025scaling}, where a single scalar feature is used to predict a binary concept.
We use 1-D probing to test whether~\methodname{} is a promising alternative for interpreting LLMs; i.e., 
whether it isolates concepts into single units, with broad coverage over human-understandable concepts of interest.
In particular, we are interested in comparing the performance of unsupervised shallow linear encoders (SAE) with our newly proposed unsupervised \textit{deep} and \textit{nonlinear} encoders (\methodname).
In addition to 1-D probing,~\autoref{subsec:dense_probe} similarly shows that dense probing performance also improves when scaling~\methodname{}.

\textbf{Method.}
We encode features with~\methodname{} via ``\metaact{}s,'' or the internal representations of the meta-model itself.
We extract~\metaact{}s at each MLP block's SwiGLU gate\footnote{
    Since our architecture mimics Llama's MLP blocks, this corresponds to the gated MLP neurons studied in prior work~\citep{choi2024automatic}: 
$\phi_i(z) = \mathrm{SiLU}\!\left(\left(w_i^{1}\right)^{\top} z\right)\cdot \left(w_i^{2}\right)^{\top} z$
}, from a single forward pass through the diffusion model.
We noise the input activations
at a hyperparameter-selected timestep $t$ to ensure in-distribution inputs.

\textbf{Setup.} For our concept set we use the 113 binary classification tasks from~\cite{kantamneni2025are}, which spans general language understanding, knowledge of geography and public figures, and topics like biology and math.
For each concept, we run probing in two stages: we first use the heuristic from~\citet{gurnee2023finding} to find a small set of candidate neurons using the train set, then fit 1-D classifiers on each candidate, selecting the best via val AUC~\cite{Bradley1997TheUO} and reporting the final test AUC.
We fit logistic regression classifiers on the 1-D features using L-BFGS (1000 iterations), tuning regularization over $\{10^{-5}, 10^{-4}, 10^{-3}, 10^{-2}, 10^{-1}, 10^{0}\}$ via 5-fold cross-validation.
Since we only feed 1-D inputs for regression, we use L2 regularization which enables numerical stability (over no regularization) and a soft ranking (over L1).
All probes are conducted on the last token activation in the sequence.
For our baselines we compare against SAEs, raw layer outputs (also the input for both SAE and~\methodname{}), and raw MLP neurons (which precedes the layer output); see~\autoref{tab:oned_probe_dim} for the number of available features per method.

\subsection{Baseline Comparison on 1-D Probes}\label{subsec:probing_oned_baselines}
We first compare~\methodname{} against competitive baselines on 1-D probing.
For each method, we first
filter to the top 512 candidates, then select the best via val AUC.
We run~\methodname{} with inputs at $t=0.1$.
As seen in~\autoref{tab:oned_probe},~\methodname{} is the best encoder for 1-D probing.
Consistent with~\citet{kantamneni2025are}, we see that SAEs are close but slightly worse in performance than the raw layer output, on Llama8B.
In fact, the raw MLP neurons are the strongest baseline, indicating that the LLM already exhibits some native disentanglement, without the help of an external encoder.
Most interestingly, the Llama1B~\methodname{} outperforms all of the Llama8B raw activations, suggesting that~\methodname{} is an encouraging alternative to LLM scaling for achieving parsimonious and human-interpretable representations.

\subsection{Scaling Behavior of 1-D Probes}\label{subsec:probing_oned}
We then investigate whether scaling improves 1-D probing performance, for  Llama1B~\methodname{}s trained on varying model sizes and data scales.
We anchor at the last checkpoint and 
filter to a single candidate per layer, then select the best via val AUC.
In~\autoref{fig:scaling:c} we visualize the results for inputs at $t=0.5$, which displays the cleanest scaling trend; see a comparison of timesteps at~\autoref{fig:scaling_probe}.
Most notably, none of the curves exhibit a plateau, meaning that allocating more compute could lead to even higher probe scores.

\subsection{Exploring Meta-Neurons}
To better understand the \metaact{}s discovered by 1-D probing, we extract maximally activating examples over a large corpus, following standard practice in automated neuron description~\cite{bills2023language,choi2024automatic}.
We take documents from the FineWeb training set, truncate them to max 64 tokens, resulting in 1M total tokens from 16k unique docs.
Since we have already localized concepts to their best \metaact{} location in the process of probing,
we can examine their consistency with their top-3 activating examples, as shown in~\autoref{tab:probe_qualitative}.
We observe that the discovered \metaact{}s exhibit consistent activation patterns, e.g.,
baseball terms for a baseball \metaact{}
or expressions of disagreement for a contradiction \metaact{}.









