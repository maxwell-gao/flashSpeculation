\section{Synthetic Tasks}
\label{sec:dataset-examples}

\begin{figure}[!t]
   \begin{bluebox}
    \raggedright
    
        \textbf{Bug Description:} The attention mechanism fails to properly normalize attention scores, leading to numerical instability and gradient explosion during training. The attention weights grow unbounded, causing immediate training divergence.
        
        \vspace{0.7em}
        \textbf{Code context:}%\\[-4pt]
        \begin{alltt}\ttfamily\footnotesize
olmo/model.py
L335: def _scaled_dot_product_attention(
L336:\hspace{1cm}self,
L337:\hspace{1cm}q: torch.Tensor,
L338:\hspace{1cm}k: torch.Tensor,
L339:\hspace{1cm}v: torch.Tensor,
L340:\hspace{1cm}attn_mask: Optional[torch.Tensor] = None,
L341:\hspace{1cm}dropout_p: float = 0.0,
L342:\hspace{1cm}is_causal: bool = False,
L343:\hspace{1cm}) -> torch.Tensor:
L344:
L345: \hspace{.8cm}attn_weights = torch.matmul(q, k.transpose(-2, -1))
L346:
L347: \hspace{.8cm}if is_causal:
L348: \hspace{1.4cm}assert attn_mask is None
L349: \hspace{1.4cm}query_len, key_len = q.shape[-2], k.shape[-2] 
        \end{alltt}
        
        \vspace{0.3em}
        
        \textbf{Target:} Given the above code context, please identify the exact location of the bug.
        
        % \\ Output your answer in the following JSON format:
        
        % % In body
        % \begin{lstlisting}[language=json, frame=none, framerule=0pt]
        % {
        %   "bug_location": "L{line_number}",
        % }
        % \end{lstlisting}
        
        \vspace{0.0em}
        {\color{blue!60}\hdashrule{\linewidth}{0.7pt}{3mm 1.2mm}}
        \vspace{-0.2cm}
        
        \textbf{Model output:} $\texttt{olmo/model.py:L345}$
        \end{bluebox}
    \caption{\label{fig:code_example}
    An example of the code bug localization synthetic task.}
\end{figure}





\begin{figure}[!t]
    \begin{redbox}
    \raggedright
    
        \textbf{Task Description:} Analyze this banking transaction log for bugs.

        \vspace{0.7em}
        
        \textbf{Initial state:} {\small $\texttt{\{"account\_A": 4000, "account\_B": 4200, "total": 8200\}}$}

        \vspace{0.7em}

        \textbf{Rules:} {1. Total money must remain constant (conservation)}

        \hspace{1.05cm}{2.\; No account can go negative}

        \hspace{1.05cm}{3.\; All calculations must be mathematically correct}
        
        \vspace{0.7em}
        \textbf{Transaction logs:}%\\[-4pt]
        \begin{alltt}\ttfamily\footnotesize
[TX001]: Transfer \$107: A=4000 → 3893, B=4200 → 4307
[TX002]: Transfer \$204: A=3893 → 3689, B=4307 → 4511 
[TX003]: Transfer \$780: A=3689 → 2909, B=4511 → 5291
[TX004]: Transfer \$2925:A=2909 → -16,  B=5291 → 8216 
\mbox{[TX005]: Transfer \$699: B=8216 → 7517, A=-16  →  683} 
        \end{alltt}
        
        \vspace{-0.3cm}

        
        
        \textbf{Possible bug types} (choose exactly one):\\
            \hspace{.2cm}-- \texttt{CALC\_ERROR}: Mathematical calculation is incorrect\\
            \hspace{.2cm}-- \texttt{NEGATIVE\_BAL}: Account balance becomes negative\\
            \hspace{.2cm}-- \texttt{LOST\_UPDATE}:  Concurrent update causes lost transaction\\
            \hspace{.2cm}-- \texttt{DUPLICATE\_TXN}:  Same transaction processed multiple times

        \vspace{0.2cm}
        
        \textbf{Target}: Please identify the bug type and location.
%         Identify the SINGLE bug if any exists.
% Respond with ONLY a JSON object:
% {"bug_type": "<type>", "bug_location": "<TX_ID>"
        % \\ Output your answer in the following JSON format:
        
        % % In body
        % \begin{lstlisting}[language=json, frame=none, framerule=0pt]
        % {
        %   "bug_location": "L{line_number}",
        % }
        % \end{lstlisting}
        
        \vspace{0.0em}
        {\color{red!60}\hdashrule{\linewidth}{0.7pt}{3mm 1.2mm}}
        \vspace{-0.2cm}
        
        \textbf{Model output:} {\small $\texttt{\{"bug\_type": NEGATIVE\_BAL, "bug\_location": TX004\}}$} 
        \end{redbox}
    \caption{\label{fig:logs_example}
    An example of the log transactions synthetic task.
    }
\end{figure}


We illustrate two representative synthetic tasks used in our study. 
Figure~\ref{fig:code_example} shows the \emph{code bug localization} task: the model receives a brief natural-language bug description together with a minimal, line-numbered code context and must return the exact file-and-line of the offending statement. In the example, the model correctly identifies the line where attention scores are computed without proper normalization (\texttt{olmo/model.py:L345}).

Figure~\ref{fig:logs_example} shows the \emph{transaction-log consistency} task: given an initial account state, a set of invariants (e.g., conservation of total funds, no negative balances), and a short sequence of transfers, the model must select a single bug type and pinpoint the first offending transaction. In the example, the model outputs \texttt{NEGATIVE\_BAL} at \texttt{TX004}, where the balance of account A becomes negative, violating the stated rules.

Together, these examples illustrate the input/output format of our synthetic tasks, the kind of structured context provided to the model, and the expected concise targets (a specific line for code or a \{\texttt{bug\_type}, \texttt{TX\_id}\} pair for logs). We use similarly formatted instances throughout our evaluation.
