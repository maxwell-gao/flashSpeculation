\section{Latency and Compute-Matched Measurements}
\label{app:latency}

\para{Setup.}
All latency numbers were measured on a single NVIDIA A100 GPU in standard inference mode.
We report wall-clock time in seconds (mean $\pm$ std) for three different context lengths.
For a given model size and context length,
we perform latency analysis based
on the amount of FLOPs, $F_{qTTT}$,
it takes to run $N_{qTTT}=32$
steps for $k=128$ on a single evaluation example.
We report the following metrics:

\begin{itemize}
    \item $N_{\text{think}}$: Number of thinking tokens
    that can be generated to match $F_{qTTT}$ FLOPs.

    \item $N_{\text{BoN}}$: Number of best-of-N trajectories
    that can be generated to match $F_{qTTT}$ FLOPs.

    \item $t_{\text{ICL}}$: Wall-clock time for a vanilla in-context pass on single example. This roughly corresponds to the prefill time.

    \item $t_{\text{think}}$: Wall-clock time to generate
    $N_{\text{think}}$ tokens, given a single example.

    \item $t_{\text{BoN}}$: The amount of time to compute best-of-N via self-consistency for $N_{\text{BoN}}$ trajectories given a single example.

    \item $t_{\text{qTTT}}$: The amount of time to perform $N_{qTTT}=32$ steps of qTTT steps with span length $k=128$
    for a single example.
\end{itemize}

\begin{table*}[t]
\centering
\small
\caption{Latency and wall-clock time comparisons given a fixed FLOP budget for Qwen3-1.7B.}
\begin{tabular}{lcccccc}
\toprule
\textbf{Context Length} & \textbf{$t_{\text{ICL}}$ (s)} & \textbf{$t_{\text{qTTT}}$ (s)} & \textbf{$t_{\text{think}}$ (s)} & \textbf{$t_{\text{BoN}}$ (s)} & \textbf{$N_{\text{think}}$} & \textbf{$N_{\text{BoN}}$}\\
\midrule
8,000     & $8.73 \pm 0.35$  & $16.92 \pm 0.68$ & $16.93 \pm 0.68$ & $16.05 \pm 0.64$ & 1{,}434 & 11 \\
32,000    & $34.93 \pm 1.40$ & $43.12 \pm 1.72$ & $43.11 \pm 1.72$ & $40.78 \pm 1.63$ & 358 & 3 \\
128,000   & $139.70 \pm 5.59$& $147.89 \pm 5.92$& $147.93 \pm 5.92$& $139.70 \pm 5.59$& 90 & 1 \\
\bottomrule
\end{tabular}
\label{tab:latency-1p7b}
\end{table*}


\begin{table*}[t]
\centering
\small
\caption{Latency and wall-clock time comparisons given a fixed FLOP budget for Qwen3-4B.}
\begin{tabular}{lcccccc}
\toprule
\textbf{Context Length} & \textbf{$t_{\text{ICL}}$ (s)} & \textbf{$t_{\text{qTTT}}$ (s)} & \textbf{$t_{\text{think}}$ (s)} & \textbf{$t_{\text{BoN}}$ (s)} & \textbf{$N_{\text{think}}$} & \textbf{$N_{\text{BoN}}$}\\
\midrule
8{,}000   & $14.61 \pm 0.58$  & $28.27 \pm 1.13$ & $28.26 \pm 1.13$ & $27.41 \pm 1.10$ & 1{,}365 & 11 \\
32{,}000  & $58.45 \pm 2.34$  & $72.11 \pm 2.88$ & $72.09 \pm 2.88$ & $68.69 \pm 2.75$ & 341 & 3 \\
128{,}000 & $233.81 \pm 9.35$ & $247.47 \pm 9.90$ & $247.41 \pm 9.90$ & $233.81 \pm 9.35$ & 85 & 1 \\
\bottomrule
\end{tabular}
\label{tab:latency-4b}
\end{table*}

\begin{table*}[t]
\centering
\small
\caption{Latency and wall-clock time comparisons given a fixed FLOP budget for Qwen3-8B.}
\begin{tabular}{lcccccc}
\toprule
\textbf{Context Length} & \textbf{$t_{\text{ICL}}$ (s)} & \textbf{$t_{\text{qTTT}}$ (s)} & \textbf{$t_{\text{think}}$ (s)} & \textbf{$t_{\text{BoN}}$ (s)} & \textbf{$N_{\text{think}}$} & \textbf{$N_{\text{BoN}}$}\\
\midrule
8{,}000   & $22.13 \pm 0.89$  & $42.61 \pm 1.70$ & $42.62 \pm 1.70$ & $41.33 \pm 1.65$ & 1{,}229 & 10 \\
32{,}000  & $88.53 \pm 3.54$  & $109.01 \pm 4.36$ & $109.00 \pm 4.36$ & $97.07 \pm 3.88$ & 307 & 2 \\
128{,}000 & $354.13 \pm 14.17$ & $374.61 \pm 14.98$ & $374.67 \pm 14.99$ & $354.13 \pm 14.17$ & 77 & 1 \\
\bottomrule
\end{tabular}
\label{tab:latency-8b}
\end{table*}


Tables~\ref{tab:latency-1p7b},~\ref{tab:latency-4b},~\ref{tab:latency-8b}
show the results of the measurements on
Qwen3-1.7B, 4B, and 8B, respectively.
% 
We find that the wall-clock time
for all three test-time compute strategies---qTTT,
thinking, and best-of-N---is quite similar.
We also note that prefilling the KV cache,
which is approximately equal to $t_{\text{ICL}}$
dominates most of the decoding time,
especially for longer sequence lengths.
This motivates the frozen K/V attention weights
in our setup, without which the prefill
would need to be recomputed with every training step.


% \para{Implementation note.}
% qTTT’s adaptation runs outside CUDA-graph capture; decoding then proceeds with the same captured graphs
% since tensor shapes and kernels are unchanged. Because only $W_Q$ is updated and the KV cache is frozen,
% we retain a single prefill and production-style paged-KV behavior across all settings.