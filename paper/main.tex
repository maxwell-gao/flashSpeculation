% --------------------------------------------------------------------------
% Your Hidden Layers Already Know How to Reason: Just Sample after Blending
%
% arXiv preprint — NeurIPS-style writing logic
% --------------------------------------------------------------------------
\documentclass[11pt]{article}

% ---------- geometry & layout ----------
\usepackage[margin=1in]{geometry}
\usepackage{setspace}
\onehalfspacing

% ---------- fonts & encoding ----------
\usepackage[T1]{fontenc}
\usepackage[utf8]{inputenc}
\usepackage{lmodern}
\usepackage{microtype}

% ---------- math ----------
\usepackage{amsmath,amssymb,amsthm}

% ---------- tables & figures ----------
\usepackage{booktabs}
\usepackage{multirow}
\usepackage{graphicx}
\usepackage{subcaption}
\usepackage[dvipsnames]{xcolor}

% ---------- algorithms ----------
\usepackage{algorithm}
\usepackage{algorithmic}

% ---------- references ----------
\usepackage[numbers,sort&compress]{natbib}
\usepackage[colorlinks,citecolor=NavyBlue,linkcolor=Maroon,urlcolor=NavyBlue]{hyperref}

% ---------- misc ----------
\usepackage{todonotes}
\usepackage{enumitem}

% ---------- custom commands ----------
% math_commands.tex — Shorthand macros for the SignFlip paper

% Greek shortcuts
\newcommand{\al}{\alpha}
\newcommand{\bt}{\beta}

% Model components
\newcommand{\lmhead}{\texttt{lm\_head}}
\newcommand{\hN}{h_N}          % final hidden state
\newcommand{\hL}{h_L}          % intermediate hidden state
\newcommand{\Ddeep}{\Delta_{\text{deep}}}  % residual h_N - h_L

% Distributions
\newcommand{\pguided}{p_{\text{guided}}}
\newcommand{\ptarget}{p_{\text{target}}}
\newcommand{\pproposal}{q_{\text{proposal}}}

% Operators
\newcommand{\softmax}{\operatorname{softmax}}
\newcommand{\logsoftmax}{\operatorname{log\text{-}softmax}}

% Placeholder for results not yet available
\newcommand{\placeholder}[1]{\textcolor{red}{\textbf{[#1]}}}

% Condition names (monospace in text)
\newcommand{\cond}[1]{\texttt{#1}}


% ==========================================================================
\title{Your Hidden Layers Already Know How to Reason:\\
Just Sample after Blending}

\author{
  Anonymous Authors \\
  \texttt{anonymous@institution.edu}
}

\date{\today}

% ==========================================================================
\begin{document}
\maketitle

% --------------------------------------------------------------------------
\begin{abstract}
% --------------------------------------------------------------------------
A growing body of work---DoLa, Contrastive Decoding, Classifier-Free
Guidance---shows that \emph{sharpening} a language model's output
distribution at inference time improves reasoning and factuality.  All of
these methods amplify the late-layer residual update ($\bt < 0$ in the
logit-blend framework), and all evaluate under greedy decoding.  The
implicit consensus: for reasoning, sharper is better.

We show this consensus is conditional on the decoding strategy.  We unify
logit-space interpolation and extrapolation under a single parameter $\bt$
that scales the deep-layer residual, and sweep $\bt$ across both signs under
greedy decoding and MCMC Power Sampling~\citep{karan_reasoning_2025}.  The
result is a \textbf{sign flip}: greedy favors $\bt < 0$ (sharpening), but
Power Sampling favors $\bt > 0$ (smoothing)---achieving 86.6\% on MATH-500
with Qwen3-4B, versus 85.4\% for standard Power Sampling and 83.0\% for
greedy.  The mechanism is geometric: the final transformer layers act
primarily as confidence refiners, and dampening their contribution softens
the target landscape that the Metropolis--Hastings chain must traverse.
Greedy needs confidence; search needs room to explore.  The method is
training-free, adds zero parameters, and fuses into the generation loop with
no extra forward passes.

\end{abstract}

% --------------------------------------------------------------------------
\section{Introduction}
\label{sec:intro}
% --------------------------------------------------------------------------

\paragraph{The sharpening consensus.}
A convergent finding across recent inference-time methods is that
\emph{sharpening} a language model's output distribution improves reasoning
and factuality.
DoLa~\citep{chuang_dola_2023} subtracts early-layer logits from late-layer
logits, amplifying the confident predictions of the deep layers.
Classifier-Free Guidance~\citep{sanchez_stay_2024} extrapolates between
conditional and unconditional logits to enforce prompt adherence.
Autocontrastive Decoding~\citep{gera_benefits_2023} penalizes the ``easy''
tokens predicted by intermediate layers.
\citet{chang_explaining_2024} prove that contrastive decoding is equivalent
to linearly extrapolating toward a hypothetical larger model.
All of these share the same geometric structure: they amplify the residual
update contributed by the final transformer layers.  And all evaluate under
\emph{greedy} (or autoregressive) decoding.

The consensus is: for reasoning, sharper is better.

\paragraph{A hidden assumption.}
But this consensus carries an implicit assumption: that the decoder makes a
single, irrevocable decision at each token position.  Greedy decoding
commits to the argmax; it benefits from a peaked distribution because the
single decision must be correct.  What happens when the decoder
\emph{searches}?

\citet{karan_reasoning_2025} showed that MCMC Power Sampling---drawing from
a sharpened distribution $p^\al$ via Metropolis--Hastings---allows base
models to match RL post-training on reasoning benchmarks without any
parameter updates.  The softmax
bottleneck~\citep{yang_breaking_2018} explains why: a standard softmax layer
can only produce a rank-$d$ log-probability matrix, while natural language
requires a high-rank matrix.  MCMC search partially bypasses this bottleneck
by exploring the sequence space rather than committing to a single path.

No prior work has asked: \emph{when the decoder is a search algorithm
rather than an argmax, does the optimal direction of logit intervention
change?}

\paragraph{The sign flip.}
We answer this question by unifying logit-space interpolation and
extrapolation under a single parameter $\bt$ that scales the deep-layer
residual:
\[
  \text{guided\_logits} = (1 - \bt) \cdot \lmhead(\hN) + \bt \cdot \lmhead(\hL)
  = \lmhead(\hN - \bt \cdot \Ddeep).
\]
Sweeping $\bt$ across both signs under greedy and MCMC decoding reveals a
\textbf{sign flip}: the optimal $\bt$ for reasoning changes sign between the
two strategies.  On MATH-500 with Qwen3-4B:
\begin{itemize}[nosep,leftmargin=*]
  \item Greedy favors $\bt < 0$ (sharpening), consistent with the
    existing literature.
  \item Power Sampling favors $\bt > 0$ (smoothing), achieving 86.6\%
    versus the 85.4\% baseline---\emph{contradicting} the sharpening
    consensus.
\end{itemize}

\paragraph{Why?}
Logit Lens studies~\citep{csordas_language_2025} and layer pruning
experiments~\citep{lu_reassessing_2025} show that the final transformer
layers act primarily as \emph{confidence refiners}: they sharpen the
distribution without introducing new semantic content.  Greedy decoding
\emph{needs} this confidence boost---the argmax must pick the right token.
But when the decoder is a Metropolis--Hastings chain targeting $p^\al$ with
$\al = 4$, the landscape is already extremely sharp.  The chain gets trapped
in narrow modes, and mixing degrades.  Dampening the confidence refinement
($\bt > 0$) softens the target landscape, giving the chain room to explore
and discover better reasoning paths.

The optimal confidence level depends on the search algorithm: greedy needs
sharpness; MCMC needs smoothness.

\paragraph{Contributions.}
\begin{enumerate}[leftmargin=*,nosep]
  \item We propose a \textbf{unified $\bt$ framework}
    (Section~\ref{sec:framework}) that subsumes DoLa, Contrastive Decoding,
    and CFG as special cases of a single logit-blend formula, exposing the
    geometric structure shared by all layer-aware inference-time
    interventions.

  \item We discover the \textbf{sign flip}: the optimal $\bt$ for reasoning
    changes sign between greedy ($\bt < 0$) and MCMC Power Sampling ($\bt >
    0$).  Logit smoothing---dismissed as harmful for reasoning by the
    contrastive decoding literature---becomes beneficial when paired with
    search (Section~\ref{sec:results}).

  \item We provide a \textbf{mechanistic explanation}
    (Section~\ref{sec:analysis}): the final layers are confidence refiners,
    not information retrievers; dampening them softens the MCMC target
    landscape and improves chain mixing.

  \item We validate the finding across \placeholder{multiple benchmarks
    (MATH-500, GSM8K, GPQA-Diamond)} and \placeholder{multiple model scales
    (Qwen3-0.6B through 8B)}.  The method is \textbf{training-free}, adds
    \textbf{zero parameters}, and is \textbf{fused} into the generation loop
    with no extra forward passes (Section~\ref{sec:implementation}).
\end{enumerate}

% --------------------------------------------------------------------------
\section{Background}
\label{sec:background}
% --------------------------------------------------------------------------

\subsection{Power Sampling via MCMC}
\label{sec:bg_power}

\citet{karan_reasoning_2025} propose sampling from the sharpened
distribution $p^\al$ ($\al > 1$) using a Metropolis--Hastings (MH)
algorithm.  Generation proceeds in blocks: a proposal sequence is generated
autoregressively at temperature $T = 1/\al$, then accepted or rejected based
on the MH ratio
\begin{equation}
  \label{eq:mh}
  \log r = \al \log p(\text{proposed}) + \log q(\text{current})
           - \al \log p(\text{current}) - \log q(\text{proposed}),
\end{equation}
where $q$ is the temperature-scaled proposal distribution.  With $\al = 4$,
the method achieves large gains on MATH-500, HumanEval, and GPQA without any
training.

\subsection{The Residual Stream and Logit Linearity}
\label{sec:bg_layer}

A transformer with $N$ layers builds its prediction through a residual
stream: $\hN = h_0 + \sum_{l=1}^{N} \delta_l$, where $\delta_l$ is the
residual update from layer $l$.  The final logits are $\lmhead(\hN)$.
Because $\lmhead$ is a linear projection, we can decompose:
\begin{equation}
  \label{eq:residual}
  \lmhead(\hN) = \lmhead(\hL) + \lmhead(\Ddeep),
  \quad \text{where} \quad \Ddeep = \hN - \hL = \sum_{l=L+1}^{N} \delta_l.
\end{equation}
This linearity is the key property that enables our framework: any linear
combination in logit space is equivalent to a linear combination in
hidden-state space.  The deep-layer residual $\Ddeep$ represents the
aggregate contribution of layers $L{+}1$ through $N$---the confidence
refinement discussed in Section~\ref{sec:refiners}.

\subsection{The Softmax Bottleneck}
\label{sec:bg_bottleneck}

\citet{yang_breaking_2018} prove that a standard softmax-based model with
embedding dimension $d$ can only produce a rank-$d$ log-probability matrix,
while natural language requires a high-rank matrix.  Their solution, Mixture
of Softmaxes, addresses this at \emph{training time}.  We address it at
\emph{inference time}: by blending logits from two different hidden states,
we create a richer distribution that is no longer a simple rank-$d$
factorization.

% --------------------------------------------------------------------------
\section{The Unified $\bt$ Framework}
\label{sec:framework}
% --------------------------------------------------------------------------

\subsection{Logit-Space Blend}
\label{sec:blend}

Given a frozen language model with $N$ layers, we define the
\emph{guided logits} at each token position as:
\begin{equation}
  \label{eq:guided}
  \boxed{
    \text{guided\_logits} = (1 - \bt) \cdot \lmhead(\hN) + \bt \cdot \lmhead(\hL)
  }
\end{equation}
where $\hL$ is the hidden state at an intermediate layer $L < N$ selected by
a probe (Section~\ref{sec:probe}), and $\bt \in \mathbb{R}$ is the blend
coefficient.

Because $\lmhead$ is linear, Equation~\ref{eq:guided} is equivalent to
\begin{equation}
  \label{eq:residual_form}
  \text{guided\_logits} = \lmhead\!\left(\hN - \bt \cdot \Ddeep\right),
\end{equation}
which reveals the geometric interpretation: $\bt$ scales the deep-layer
residual contribution $\Ddeep = \hN - \hL$.

\begin{table}[h]
\centering
\caption{The unified $\bt$ framework: three regimes.}
\label{tab:regimes}
\begin{tabular}{@{}lcll@{}}
\toprule
$\bt$ regime & Effect on $\Ddeep$ & Distribution & Prior work \\
\midrule
$\bt < 0$ (extrapolation) & Amplify & Sharper, more confident &
  DoLa~\citep{chuang_dola_2023}, CFG~\citep{sanchez_stay_2024} \\
$\bt = 0$ & Identity & Standard decoding & Baseline \\
$\bt > 0$ (interpolation) & Dampen & Smoother, less confident &
  \textbf{This work} \\
\bottomrule
\end{tabular}
\end{table}

\subsection{Relationship to Prior Work}
\label{sec:prior_map}

\paragraph{DoLa ($\bt \approx -1$, dynamic layer).}
DoLa~\citep{chuang_dola_2023} computes $\lmhead(\hN) - \lmhead(\hL)$ with
a dynamically selected contrast layer, equivalent to $\bt = -1$ in our
framework.  It is evaluated under greedy decoding.

\paragraph{Contrastive Decoding ($\bt < 0$, two models).}
\citet{chang_explaining_2024} prove that contrastive decoding (expert minus
amateur logits) can be viewed as linear extrapolation toward a hypothetical
larger model.  When both models share a single backbone (as in our
framework), this reduces to logit extrapolation with $\bt < 0$.

\paragraph{Classifier-Free Guidance ($\bt \in [-2, -1]$).}
CFG~\citep{sanchez_stay_2024} interpolates between conditional and
unconditional logits with a guidance scale $\gamma > 1$, equivalent to
$\bt = 1 - \gamma < 0$ when the ``unconditional'' signal is an intermediate
layer's prediction.

\paragraph{This work: $\bt > 0$ under MCMC.}
We are the first to study \emph{positive} $\bt$ (interpolation /
smoothing) for reasoning tasks, and the first to sweep $\bt$ across
both signs under MCMC sampling.

\subsection{Combining with MCMC Power Sampling}
\label{sec:combine}

When $\bt \neq 0$, the MH target distribution becomes $\pguided^\al$:
\begin{equation}
  \label{eq:guided_mh}
  \log r = \al \log \pguided(\text{proposed}) + \log q(\text{current})
           - \al \log \pguided(\text{current}) - \log q(\text{proposed}).
\end{equation}
The proposal $q$ remains the standard temperature-scaled distribution
(unchanged from~\citet{karan_reasoning_2025}).  Only the
\emph{target evaluation} changes: it uses the blended logits from
Equation~\ref{eq:guided}.

% --------------------------------------------------------------------------
\section{Experimental Setup}
\label{sec:setup}
% --------------------------------------------------------------------------

\subsection{Models}
\label{sec:models}

We evaluate on the Qwen3 model family in bfloat16 precision with Flash
Attention 2:

\begin{table}[h]
\centering
\caption{Model configurations. The blend layer $L$ is selected by the probe
  (Section~\ref{sec:probe}); $N{-}3$ is used as a heuristic validated by
  the probe.}
\label{tab:models}
\begin{tabular}{@{}lccc@{}}
\toprule
Model & Layers ($N$) & Blend layer ($L$) & Parameters \\
\midrule
Qwen3-0.6B & 28 & \placeholder{25} & 0.6B \\
Qwen3-1.7B & 28 & \placeholder{25} & 1.7B \\
Qwen3-4B   & 36 & 33               & 4B \\
Qwen3-8B   & 36 & \placeholder{33} & 8B \\
\bottomrule
\end{tabular}
\end{table}

\subsection{Benchmarks}
\label{sec:benchmarks}

\begin{itemize}[nosep,leftmargin=*]
  \item \textbf{MATH-500}~(500 competition math problems): the primary benchmark,
    following~\citet{karan_reasoning_2025}.
  \item \textbf{GSM8K}~(1319 grade-school math problems): tests whether the
    effect transfers to easier arithmetic reasoning.
  \item \textbf{GPQA-Diamond}~(198 graduate-level science questions):
    multiple-choice format testing scientific reasoning.
\end{itemize}

\subsection{Probe-Based Layer Selection}
\label{sec:probe}

Rather than fixing the blend layer $L$ via a heuristic (e.g., $N{-}3$), we
run a lightweight probe: for each layer $l \in \{0, \ldots, N{-}1\}$, we
compute $\lmhead(h_l)$ on a held-out sample of 50 examples from the target
benchmark and measure the median rank of the gold next-token.  The
recommended $L$ is the non-final layer with the lowest median rank.

For Qwen3-4B on MATH-500, the probe selects layer 33 (median rank 12),
confirming the choice from our earlier diagnostic
experiments.\footnote{The Phase~0 diagnostic report showed layer~33 achieves
rank~12 on the gold token, while the next-best layer~25 has rank~21,981.}

\subsection{Implementation: Fused Blend}
\label{sec:implementation}

A na\"ive implementation would require a separate forward pass to extract
$\hL$ for the MH target evaluation.  We instead \emph{fuse} the blend into
the existing generation call: \texttt{model.generate(...,
output\_hidden\_states=True)} already runs the full model and returns all
hidden states at each decoding step.  We extract $h_L$ from
\texttt{output.hidden\_states[L+1]} and compute the blended logits in-place,
adding \textbf{zero extra forward passes}.

\subsection{Conditions and Hyperparameters}
\label{sec:conditions}

\begin{table}[h]
\centering
\caption{Experimental conditions.  All conditions are training-free and
  parameter-free.}
\label{tab:conditions}
\begin{tabular}{@{}llccc@{}}
\toprule
Condition & Base distribution & Sampling & MCMC & Extra cost \\
\midrule
\cond{greedy}       & $p$                 & argmax      & No  & --- \\
\cond{ps}           & $p^\al$             & MH          & Yes & --- \\
\cond{blend\_greedy} & $\pguided$          & argmax      & No  & 0 \\
\cond{blend\_ps}     & $\pguided^\al$      & MH          & Yes & 0 \\
\bottomrule
\end{tabular}
\end{table}

Default hyperparameters: $\al = 4.0$, $\bt = 0.05$, MCMC steps $= 10$,
blocks $= 16$, \texttt{max\_new\_tokens} $= 3072$.

% --------------------------------------------------------------------------
\section{Results}
\label{sec:results}
% --------------------------------------------------------------------------

\subsection{The SignFlip: Main Result}
\label{sec:signflip}

\begin{table}[h]
\centering
\caption{MATH-500 accuracy (\%) with Qwen3-4B.  \textbf{Bold} marks the
  best result per decoding strategy.  The sign flip is visible: $\bt > 0$
  hurts greedy but helps MCMC.}
\label{tab:main}
\begin{tabular}{@{}lcc@{}}
\toprule
Condition & Accuracy (\%) & $\Delta$ vs.\ baseline \\
\midrule
\cond{greedy} ($\bt = 0$)              & 83.0 & --- \\
\cond{blend\_greedy} ($\bt = 0.05$)    & 82.4 & $-0.6$ \\
\midrule
\cond{ps} ($\bt = 0$, $\al = 4$)      & 85.4 & --- \\
\cond{blend\_ps} ($\bt = 0.05$, $\al = 4$) & \textbf{86.6} & $+1.2$ \\
\bottomrule
\end{tabular}
\end{table}

The interaction is superlinear: logit smoothing \emph{hurts} greedy
($-0.6$\,pp) but \emph{helps} MCMC ($+1.2$\,pp).  The combined method
achieves 86.6\%, outperforming both the greedy baseline (83.0\%) and the
Power Sampling baseline (85.4\%).

\begin{figure}[t]
\centering
\missingfigure[figwidth=0.9\textwidth]{%
  $\bt$ sweep figure: x-axis = $\bt$ from $-0.20$ to $+0.20$,
  y-axis = accuracy (\%).
  Two curves: greedy (monotonically decreasing as $\bt$ increases)
  and Power Sampling (increasing as $\bt$ increases).
  The curves cross near $\bt = 0$---the SignFlip.%
}
\caption{The SignFlip.  Accuracy on MATH-500 (Qwen3-4B) as a function of
  $\bt$ under greedy decoding and Power Sampling ($\al = 4$).  The optimal
  $\bt$ flips sign: greedy favors $\bt < 0$ (sharpening), while MCMC favors
  $\bt > 0$ (smoothing).}
\label{fig:signflip}
\end{figure}

\subsection{$\bt$ Sweep}
\label{sec:sweep}

\placeholder{Full $\bt$ sweep results with $\bt \in \{-0.20, -0.10, -0.05,
0, +0.05, +0.10, +0.20\}$ under both greedy and Power Sampling.  We expect
the greedy curve to peak at a small negative $\bt$ and the PS curve to peak
at a moderate positive $\bt$, crossing near $\bt = 0$.}

\subsection{Model Scaling}
\label{sec:scaling}

\begin{table}[h]
\centering
\caption{MATH-500 accuracy (\%) across model scales at the best $\bt$
  configuration.}
\label{tab:scaling}
\begin{tabular}{@{}lcccc@{}}
\toprule
Model & \cond{greedy} & \cond{ps} & \cond{blend\_ps} & $\Delta$ (ps $\to$ blend\_ps) \\
\midrule
Qwen3-0.6B & \placeholder{--} & \placeholder{--} & \placeholder{--} & \placeholder{--} \\
Qwen3-1.7B & \placeholder{--} & \placeholder{--} & \placeholder{--} & \placeholder{--} \\
Qwen3-4B   & 83.0 & 85.4 & \textbf{86.6} & $+1.2$ \\
Qwen3-8B   & \placeholder{--} & \placeholder{--} & \placeholder{--} & \placeholder{--} \\
\bottomrule
\end{tabular}
\end{table}

\placeholder{Analysis of whether the sign flip persists across model scales,
and whether the optimal $\bt$ shifts with model size.}

\subsection{Benchmark Diversity}
\label{sec:benchmarks_results}

\begin{table}[h]
\centering
\caption{Accuracy (\%) across benchmarks with Qwen3-4B at $\bt = 0.05$,
  $\al = 4$.}
\label{tab:benchmarks}
\begin{tabular}{@{}lccc@{}}
\toprule
Benchmark & \cond{greedy} & \cond{ps} & \cond{blend\_ps} \\
\midrule
MATH-500       & 83.0 & 85.4 & \textbf{86.6} \\
GSM8K          & \placeholder{--} & \placeholder{--} & \placeholder{--} \\
GPQA-Diamond   & \placeholder{--} & \placeholder{--} & \placeholder{--} \\
\bottomrule
\end{tabular}
\end{table}

\placeholder{Discussion of whether the sign flip generalizes beyond
mathematical reasoning to scientific reasoning (GPQA) and simpler arithmetic
(GSM8K).}

\subsection{Ablations}
\label{sec:ablations}

\paragraph{Blend layer position.}
\placeholder{Sweep over different blend layers $L$ to validate the probe
selection and test sensitivity.}

\paragraph{$\al$ sensitivity.}
\placeholder{Test $\al \in \{2, 4, 8\}$ to verify robustness of the sign
flip across sharpening strengths.}

% --------------------------------------------------------------------------
\section{Analysis: Why Does Smoothing Help MCMC?}
\label{sec:analysis}
% --------------------------------------------------------------------------

The sign flip raises a natural question: why does the same intervention that
hurts greedy \emph{help} MCMC?  We argue that the answer lies in the
functional role of the final transformer layers and how that role interacts
with different decoding strategies.

\subsection{The Final Layers as Confidence Refiners}
\label{sec:refiners}

Logit Lens studies---projecting each layer's hidden state through the
language model head---consistently show that the ``concept'' of the correct
token is resolved well before the final layer, typically by layer $2N/3$.
The remaining layers primarily \emph{boost confidence}: they increase the
logit magnitude of the already-leading candidate relative to distractors,
reducing entropy without introducing new semantic
content~\citep{csordas_language_2025}.  Layer pruning
experiments~\citep{lu_reassessing_2025} corroborate this: removing the final
10\% of layers barely degrades knowledge tasks (MMLU) but collapses complex
reasoning (AIME)---suggesting that these layers refine the \emph{precision}
of the output, not the \emph{identity} of the answer.

In the unified $\bt$ framework, the deep-layer residual $\Ddeep = \hN -
\hL$ is precisely this confidence-refinement vector.  The sign of $\bt$
controls whether we amplify it ($\bt < 0$: extrapolation, sharper) or dampen
it ($\bt > 0$: interpolation, smoother).

\subsection{Why Greedy Needs Sharpness}

Greedy decoding commits to the single highest-probability token at each
step.  A small perturbation that shifts the argmax from the correct to an
incorrect token causes cascading errors in autoregressive generation.
Amplifying the confidence-refinement vector ($\bt < 0$) makes the argmax
more robust: the correct token's logit margin over distractors widens,
reducing the probability of a wrong commitment.  This explains why DoLa,
Contrastive Decoding, and CFG all improve greedy reasoning: they manually
accelerate the model's native confidence-refinement process.

\subsection{Why MCMC Needs Smoothness}

Power Sampling targets $\pguided^\al$ with $\al = 4$.  Raising the
distribution to a power of~4 already produces an extremely sharp landscape.
The Metropolis--Hastings chain must traverse this landscape to find good
reasoning paths, but sharp landscapes have deep, narrow modes separated by
probability deserts.  The chain gets trapped: proposals that deviate from the
current mode are overwhelmingly rejected.

Dampening the confidence-refinement vector ($\bt > 0$) softens the modes
\emph{before} the power operation: $\pguided^\al$ has wider basins of
attraction than $p^\al$, because the per-token logit margins are smaller.
The chain can transition between modes more easily, exploring a larger
region of the sequence space and discovering reasoning paths that standard
Power Sampling misses.

\subsection{Acceptance Rate vs.\ Sample Quality}

In our experiments on the 104-problem subset, the MH acceptance rate for
\cond{blend\_ps} (50.4\%) is lower than \cond{ps} (55.8\%)---seemingly
contradicting the ``better mixing'' argument.  But this comparison is
between chains targeting \emph{different} distributions.  The guided target
$\pguided^\al$ concentrates mass more effectively on correct reasoning
paths; proposals drawn from $q$ (which targets the standard distribution)
are further from the guided mode and thus rejected more often.  A lower
acceptance rate with a better target is a sign of a \emph{higher-quality
baseline}, not of a stuck chain.  The relevant metric is the
\emph{effective sample quality}---measured by downstream accuracy---which
improves from 72.1\% to 77.9\% on this subset.

% --------------------------------------------------------------------------
\section{Related Work}
\label{sec:related}
% --------------------------------------------------------------------------

\paragraph{Layer-aware decoding.}
DoLa~\citep{chuang_dola_2023} contrasts late vs.\ early layers to improve
factuality.  \citet{gera_benefits_2023} use autocontrastive layer signals to
reduce degeneration in open-ended generation.
\citet{chen_-context_2024} discover that hallucinations correlate with
in-context activation sharpness across layers, and use this signal for
constrained decoding.  All operate in the $\bt < 0$ regime under greedy
decoding.

\paragraph{Contrastive and guided decoding.}
Classifier-Free Guidance~\citep{sanchez_stay_2024} applies
conditional--unconditional extrapolation ($\bt < 0$) to improve
prompt adherence in language generation.  \citet{chang_explaining_2024}
prove that contrastive decoding is equivalent to linearly extrapolating the
logits of a hypothetical larger model.
ThinkLogit~\citep{zhang_logit_2025} applies the same logit-arithmetic
structure \emph{across models}: the delta between a small reasoning model
and its base is added to a large non-reasoning model, eliciting long
chain-of-thought without training the large model.  This is mathematically
isomorphic to our $\bt$ framework with $\bt < 0$, but between models rather
than layers.  REAL Sampling~\citep{chang_real_2025} dynamically adjusts the
sampling threshold based on extrapolated entropy from a model-size scaling
series.  SC-MCTS*~\citep{gao_interpretable_2024} applies contrastive
decoding as a reward signal within Monte Carlo Tree Search.

\paragraph{Inference-time compute scaling.}
Power Sampling~\citep{karan_reasoning_2025} shows that MCMC from $p^\al$
matches RL post-training on reasoning benchmarks.
\citet{bansal_smaller_2024} study compute-optimal sampling strategies and
find that weaker models can be more cost-effective data generators.
\citet{kang_scalable_nodate} propose self-certainty as a scalable
alternative to Best-of-N selection.  \citet{song_evaluation_nodate} argue
that LLM evaluation must account for non-determinism.
Min-p sampling~\citep{minh_turning_2024} introduces dynamic truncation that
scales the threshold with the top token's probability, maintaining coherence
at high temperatures where nucleus sampling degrades.  Like our work, min-p
highlights that the interaction between distribution shape and sampling
strategy is critical: the optimal truncation policy depends on how peaked
the distribution is.  Our contribution is orthogonal---we modify the
distribution itself rather than the truncation rule, and show that the
optimal direction of modification depends on the search algorithm.

\paragraph{Internal representations.}
Patchscopes~\citep{ghandeharioun_patchscopes_2024} provides a unifying
framework for inspecting hidden representations by patching them into
the model's own computation.  \citet{kuribayashi_large_2025} show that
internal layers of larger LLMs align better with human reading behavior
than final layers---suggesting that intermediate representations carry
cognitively relevant information.  \citet{csordas_language_2025} find
that deeper models spread the same computations across more layers
rather than composing new features, and \citet{lu_reassessing_2025}
show that pruning final layers has minimal impact---both supporting
our hypothesis that deep-layer residuals carry refinement rather than
core knowledge.

% --------------------------------------------------------------------------
\section{Discussion and Future Work}
\label{sec:discussion}
% --------------------------------------------------------------------------

\paragraph{Limitations.}
Our current results are on MATH-500 with Qwen3-4B at a single $\bt$
value.  \placeholder{The full Phase~1 evaluation across multiple
benchmarks, model scales, and $\bt$ values is in progress.}
The method requires MCMC sampling, which is computationally more expensive
than greedy decoding (though it adds no cost \emph{on top of} standard
Power Sampling).  The probe-based layer selection adds a small one-time
overhead per model.

\paragraph{Toward trainable guided decoding.}
This work uses a fixed intermediate layer as the blend source---a single
linear projection $\lmhead(\hL)$.  A natural extension is to replace this
with a \emph{trainable readout} that aggregates multiple intermediate
layers, potentially combined with test-time training to produce a
context-specialized blend source that adapts per input.

% --------------------------------------------------------------------------
\section{Conclusion}
\label{sec:conclusion}
% --------------------------------------------------------------------------

The prevailing consensus in inference-time intervention holds that
sharpening (extrapolation, $\bt < 0$) is universally beneficial for
reasoning.  We showed that this consensus is conditional on a hidden
assumption: greedy decoding.  By unifying logit-space interpolation and
extrapolation under a single parameter $\bt$ and crossing it with the
decoding strategy, we discovered the \emph{sign flip}: MCMC Power Sampling
favors the opposite direction---smoothing ($\bt > 0$)---because it needs a
traversable landscape, not a confident point estimate.

The mechanistic explanation is clean: the final transformer layers are
confidence refiners.  Greedy decoding needs their refinement; MCMC search
does not.  Dampening the deep-layer residual ($\bt > 0$) softens the
$p^\al$ target, improves chain mixing, and leads to better reasoning
outcomes.

The method is training-free, adds zero parameters, fuses into the generation
loop, and achieves 86.6\% on MATH-500 with Qwen3-4B---surpassing both
greedy decoding (83.0\%) and standard Power Sampling (85.4\%).  The broader
insight---\emph{the optimal confidence level depends on the search
strategy}---suggests that inference-time scaling methods should co-design
their distribution modifications and their search algorithms, rather than
optimizing each in isolation.

% --------------------------------------------------------------------------
% References
% --------------------------------------------------------------------------
\bibliographystyle{plainnat}
\bibliography{reference}

\end{document}
